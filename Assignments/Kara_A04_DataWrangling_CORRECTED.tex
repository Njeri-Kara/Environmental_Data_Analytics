\documentclass[]{article}
\usepackage{lmodern}
\usepackage{amssymb,amsmath}
\usepackage{ifxetex,ifluatex}
\usepackage{fixltx2e} % provides \textsubscript
\ifnum 0\ifxetex 1\fi\ifluatex 1\fi=0 % if pdftex
  \usepackage[T1]{fontenc}
  \usepackage[utf8]{inputenc}
\else % if luatex or xelatex
  \ifxetex
    \usepackage{mathspec}
  \else
    \usepackage{fontspec}
  \fi
  \defaultfontfeatures{Ligatures=TeX,Scale=MatchLowercase}
\fi
% use upquote if available, for straight quotes in verbatim environments
\IfFileExists{upquote.sty}{\usepackage{upquote}}{}
% use microtype if available
\IfFileExists{microtype.sty}{%
\usepackage{microtype}
\UseMicrotypeSet[protrusion]{basicmath} % disable protrusion for tt fonts
}{}
\usepackage[margin=2.54cm]{geometry}
\usepackage{hyperref}
\hypersetup{unicode=true,
            pdftitle={Assignment 4: Data Wrangling},
            pdfauthor={Njeri Kara},
            pdfborder={0 0 0},
            breaklinks=true}
\urlstyle{same}  % don't use monospace font for urls
\usepackage{color}
\usepackage{fancyvrb}
\newcommand{\VerbBar}{|}
\newcommand{\VERB}{\Verb[commandchars=\\\{\}]}
\DefineVerbatimEnvironment{Highlighting}{Verbatim}{commandchars=\\\{\}}
% Add ',fontsize=\small' for more characters per line
\usepackage{framed}
\definecolor{shadecolor}{RGB}{248,248,248}
\newenvironment{Shaded}{\begin{snugshade}}{\end{snugshade}}
\newcommand{\KeywordTok}[1]{\textcolor[rgb]{0.13,0.29,0.53}{\textbf{#1}}}
\newcommand{\DataTypeTok}[1]{\textcolor[rgb]{0.13,0.29,0.53}{#1}}
\newcommand{\DecValTok}[1]{\textcolor[rgb]{0.00,0.00,0.81}{#1}}
\newcommand{\BaseNTok}[1]{\textcolor[rgb]{0.00,0.00,0.81}{#1}}
\newcommand{\FloatTok}[1]{\textcolor[rgb]{0.00,0.00,0.81}{#1}}
\newcommand{\ConstantTok}[1]{\textcolor[rgb]{0.00,0.00,0.00}{#1}}
\newcommand{\CharTok}[1]{\textcolor[rgb]{0.31,0.60,0.02}{#1}}
\newcommand{\SpecialCharTok}[1]{\textcolor[rgb]{0.00,0.00,0.00}{#1}}
\newcommand{\StringTok}[1]{\textcolor[rgb]{0.31,0.60,0.02}{#1}}
\newcommand{\VerbatimStringTok}[1]{\textcolor[rgb]{0.31,0.60,0.02}{#1}}
\newcommand{\SpecialStringTok}[1]{\textcolor[rgb]{0.31,0.60,0.02}{#1}}
\newcommand{\ImportTok}[1]{#1}
\newcommand{\CommentTok}[1]{\textcolor[rgb]{0.56,0.35,0.01}{\textit{#1}}}
\newcommand{\DocumentationTok}[1]{\textcolor[rgb]{0.56,0.35,0.01}{\textbf{\textit{#1}}}}
\newcommand{\AnnotationTok}[1]{\textcolor[rgb]{0.56,0.35,0.01}{\textbf{\textit{#1}}}}
\newcommand{\CommentVarTok}[1]{\textcolor[rgb]{0.56,0.35,0.01}{\textbf{\textit{#1}}}}
\newcommand{\OtherTok}[1]{\textcolor[rgb]{0.56,0.35,0.01}{#1}}
\newcommand{\FunctionTok}[1]{\textcolor[rgb]{0.00,0.00,0.00}{#1}}
\newcommand{\VariableTok}[1]{\textcolor[rgb]{0.00,0.00,0.00}{#1}}
\newcommand{\ControlFlowTok}[1]{\textcolor[rgb]{0.13,0.29,0.53}{\textbf{#1}}}
\newcommand{\OperatorTok}[1]{\textcolor[rgb]{0.81,0.36,0.00}{\textbf{#1}}}
\newcommand{\BuiltInTok}[1]{#1}
\newcommand{\ExtensionTok}[1]{#1}
\newcommand{\PreprocessorTok}[1]{\textcolor[rgb]{0.56,0.35,0.01}{\textit{#1}}}
\newcommand{\AttributeTok}[1]{\textcolor[rgb]{0.77,0.63,0.00}{#1}}
\newcommand{\RegionMarkerTok}[1]{#1}
\newcommand{\InformationTok}[1]{\textcolor[rgb]{0.56,0.35,0.01}{\textbf{\textit{#1}}}}
\newcommand{\WarningTok}[1]{\textcolor[rgb]{0.56,0.35,0.01}{\textbf{\textit{#1}}}}
\newcommand{\AlertTok}[1]{\textcolor[rgb]{0.94,0.16,0.16}{#1}}
\newcommand{\ErrorTok}[1]{\textcolor[rgb]{0.64,0.00,0.00}{\textbf{#1}}}
\newcommand{\NormalTok}[1]{#1}
\usepackage{longtable,booktabs}
\usepackage{graphicx,grffile}
\makeatletter
\def\maxwidth{\ifdim\Gin@nat@width>\linewidth\linewidth\else\Gin@nat@width\fi}
\def\maxheight{\ifdim\Gin@nat@height>\textheight\textheight\else\Gin@nat@height\fi}
\makeatother
% Scale images if necessary, so that they will not overflow the page
% margins by default, and it is still possible to overwrite the defaults
% using explicit options in \includegraphics[width, height, ...]{}
\setkeys{Gin}{width=\maxwidth,height=\maxheight,keepaspectratio}
\IfFileExists{parskip.sty}{%
\usepackage{parskip}
}{% else
\setlength{\parindent}{0pt}
\setlength{\parskip}{6pt plus 2pt minus 1pt}
}
\setlength{\emergencystretch}{3em}  % prevent overfull lines
\providecommand{\tightlist}{%
  \setlength{\itemsep}{0pt}\setlength{\parskip}{0pt}}
\setcounter{secnumdepth}{0}
% Redefines (sub)paragraphs to behave more like sections
\ifx\paragraph\undefined\else
\let\oldparagraph\paragraph
\renewcommand{\paragraph}[1]{\oldparagraph{#1}\mbox{}}
\fi
\ifx\subparagraph\undefined\else
\let\oldsubparagraph\subparagraph
\renewcommand{\subparagraph}[1]{\oldsubparagraph{#1}\mbox{}}
\fi

%%% Use protect on footnotes to avoid problems with footnotes in titles
\let\rmarkdownfootnote\footnote%
\def\footnote{\protect\rmarkdownfootnote}

%%% Change title format to be more compact
\usepackage{titling}

% Create subtitle command for use in maketitle
\newcommand{\subtitle}[1]{
  \posttitle{
    \begin{center}\large#1\end{center}
    }
}

\setlength{\droptitle}{-2em}

  \title{Assignment 4: Data Wrangling}
    \pretitle{\vspace{\droptitle}\centering\huge}
  \posttitle{\par}
    \author{Njeri Kara}
    \preauthor{\centering\large\emph}
  \postauthor{\par}
    \date{}
    \predate{}\postdate{}
  

\begin{document}
\maketitle

\subsection{OVERVIEW}\label{overview}

This exercise accompanies the lessons in Environmental Data Analytics
(ENV872L) on data wrangling.

\subsection{Directions}\label{directions}

\begin{enumerate}
\def\labelenumi{\arabic{enumi}.}
\tightlist
\item
  Change ``Student Name'' on line 3 (above) with your name.
\item
  Use the lesson as a guide. It contains code that can be modified to
  complete the assignment.
\item
  Work through the steps, \textbf{creating code and output} that fulfill
  each instruction.
\item
  Be sure to \textbf{answer the questions} in this assignment document.
  Space for your answers is provided in this document and is indicated
  by the ``\textgreater{}'' character. If you need a second paragraph be
  sure to start the first line with ``\textgreater{}''. You should
  notice that the answer is highlighted in green by RStudio.
\item
  When you have completed the assignment, \textbf{Knit} the text and
  code into a single PDF file. You will need to have the correct
  software installed to do this (see Software Installation Guide) Press
  the \texttt{Knit} button in the RStudio scripting panel. This will
  save the PDF output in your Assignments folder.
\item
  After Knitting, please submit the completed exercise (PDF file) to the
  dropbox in Sakai. Please add your last name into the file name (e.g.,
  ``Salk\_A04\_DataWrangling.pdf'') prior to submission.
\end{enumerate}

The completed exercise is due on Thursday, 7 February, 2019 before class
begins.

\subsection{Set up your session}\label{set-up-your-session}

\begin{enumerate}
\def\labelenumi{\arabic{enumi}.}
\item
  Check your working directory, load the \texttt{tidyverse} package, and
  upload all four raw data files associated with the EPA Air dataset.
  See the README file for the EPA air datasets for more information
  (especially if you have not worked with air quality data previously).
\item
  Generate a few lines of code to get to know your datasets (basic data
  summaries, etc.).
\end{enumerate}

\begin{Shaded}
\begin{Highlighting}[]
\CommentTok{#1}
\CommentTok{#Setting the working directory}
\KeywordTok{setwd}\NormalTok{(}\StringTok{"C:/Users/jerik/OneDrive/Documents/Spring 2019 Semenster/Environmental Data Analytics/EDA_R_Work/EDA"}\NormalTok{)}
\CommentTok{#Confirming that it is the correct working directory}
\KeywordTok{getwd}\NormalTok{()}
\end{Highlighting}
\end{Shaded}

\begin{verbatim}
## [1] "C:/Users/jerik/OneDrive/Documents/Spring 2019 Semenster/Environmental Data Analytics/EDA_R_Work/EDA"
\end{verbatim}

\begin{Shaded}
\begin{Highlighting}[]
\CommentTok{#Loading necessary packages}
\KeywordTok{library}\NormalTok{(tidyverse)}
\end{Highlighting}
\end{Shaded}

\begin{verbatim}
## -- Attaching packages ---------------------------------------------------------------------------------------- tidyverse 1.2.1 --
\end{verbatim}

\begin{verbatim}
## v ggplot2 3.0.0     v purrr   0.2.5
## v tibble  1.4.2     v dplyr   0.7.6
## v tidyr   0.8.2     v stringr 1.3.1
## v readr   1.1.1     v forcats 0.3.0
\end{verbatim}

\begin{verbatim}
## -- Conflicts ------------------------------------------------------------------------------------------- tidyverse_conflicts() --
## x dplyr::filter() masks stats::filter()
## x dplyr::lag()    masks stats::lag()
\end{verbatim}

\begin{Shaded}
\begin{Highlighting}[]
\KeywordTok{library}\NormalTok{(lubridate)}
\end{Highlighting}
\end{Shaded}

\begin{verbatim}
## 
## Attaching package: 'lubridate'
\end{verbatim}

\begin{verbatim}
## The following object is masked from 'package:base':
## 
##     date
\end{verbatim}

\begin{Shaded}
\begin{Highlighting}[]
\KeywordTok{library}\NormalTok{(knitr)}

\CommentTok{#Uploading the four raw datafiles associated with EPA Air dataset.}
\NormalTok{NC.}\FloatTok{03.2017}\NormalTok{.raw.data <-}\StringTok{ }\KeywordTok{read.csv}\NormalTok{(}\StringTok{"./Data/Raw/EPAair_O3_NC2017_raw.csv"}\NormalTok{)}
\NormalTok{NC.}\FloatTok{03.2018}\NormalTok{.raw.data <-}\StringTok{ }\KeywordTok{read.csv}\NormalTok{(}\StringTok{"./Data/Raw/EPAair_O3_NC2018_raw.csv"}\NormalTok{)}
\NormalTok{NC.PM25.}\FloatTok{2017.}\NormalTok{raw.data <-}\StringTok{ }\KeywordTok{read.csv}\NormalTok{(}\StringTok{"./Data/Raw/EPAair_PM25_NC2017_raw.csv"}\NormalTok{)}
\NormalTok{NC.PM25.}\FloatTok{2018.}\NormalTok{raw.data <-}\StringTok{ }\KeywordTok{read.csv}\NormalTok{(}\StringTok{"./Data/Raw/EPAair_PM25_NC2018_raw.csv"}\NormalTok{)}

\CommentTok{#2}
\CommentTok{#Getting to know NC.03.2017 data}
\KeywordTok{dim}\NormalTok{(NC.}\FloatTok{03.2017}\NormalTok{.raw.data) }\CommentTok{#shows number of rows and columns in the dataset}
\end{Highlighting}
\end{Shaded}

\begin{verbatim}
## [1] 10219    20
\end{verbatim}

\begin{Shaded}
\begin{Highlighting}[]
\KeywordTok{str}\NormalTok{(NC.}\FloatTok{03.2017}\NormalTok{.raw.data) }\CommentTok{#shows the names and class of each variable and a sample of its values}
\end{Highlighting}
\end{Shaded}

\begin{verbatim}
## 'data.frame':    10219 obs. of  20 variables:
##  $ Date                                : Factor w/ 364 levels "1/1/17","1/10/17",..: 151 162 173 176 177 178 179 180 181 152 ...
##  $ Source                              : Factor w/ 1 level "AQS": 1 1 1 1 1 1 1 1 1 1 ...
##  $ Site.ID                             : int  370030005 370030005 370030005 370030005 370030005 370030005 370030005 370030005 370030005 370030005 ...
##  $ POC                                 : int  1 1 1 1 1 1 1 1 1 1 ...
##  $ Daily.Max.8.hour.Ozone.Concentration: num  0.041 0.046 0.046 0.046 0.046 0.048 0.047 0.053 0.056 0.048 ...
##  $ UNITS                               : Factor w/ 1 level "ppm": 1 1 1 1 1 1 1 1 1 1 ...
##  $ DAILY_AQI_VALUE                     : int  38 43 43 43 43 44 44 49 54 44 ...
##  $ Site.Name                           : Factor w/ 40 levels "","Beaufort",..: 35 35 35 35 35 35 35 35 35 35 ...
##  $ DAILY_OBS_COUNT                     : int  17 17 17 17 17 17 17 17 17 17 ...
##  $ PERCENT_COMPLETE                    : int  100 100 100 100 100 100 100 100 100 100 ...
##  $ AQS_PARAMETER_CODE                  : int  44201 44201 44201 44201 44201 44201 44201 44201 44201 44201 ...
##  $ AQS_PARAMETER_DESC                  : Factor w/ 1 level "Ozone": 1 1 1 1 1 1 1 1 1 1 ...
##  $ CBSA_CODE                           : int  25860 25860 25860 25860 25860 25860 25860 25860 25860 25860 ...
##  $ CBSA_NAME                           : Factor w/ 17 levels "","Asheville, NC",..: 9 9 9 9 9 9 9 9 9 9 ...
##  $ STATE_CODE                          : int  37 37 37 37 37 37 37 37 37 37 ...
##  $ STATE                               : Factor w/ 1 level "North Carolina": 1 1 1 1 1 1 1 1 1 1 ...
##  $ COUNTY_CODE                         : int  3 3 3 3 3 3 3 3 3 3 ...
##  $ COUNTY                              : Factor w/ 32 levels "Alexander","Avery",..: 1 1 1 1 1 1 1 1 1 1 ...
##  $ SITE_LATITUDE                       : num  35.9 35.9 35.9 35.9 35.9 ...
##  $ SITE_LONGITUDE                      : num  -81.2 -81.2 -81.2 -81.2 -81.2 ...
\end{verbatim}

\begin{Shaded}
\begin{Highlighting}[]
\KeywordTok{head}\NormalTok{(NC.}\FloatTok{03.2017}\NormalTok{.raw.data) }\CommentTok{#shows the first six observations in the dataset}
\end{Highlighting}
\end{Shaded}

\begin{verbatim}
##     Date Source   Site.ID POC Daily.Max.8.hour.Ozone.Concentration UNITS
## 1 3/1/17    AQS 370030005   1                                0.041   ppm
## 2 3/2/17    AQS 370030005   1                                0.046   ppm
## 3 3/3/17    AQS 370030005   1                                0.046   ppm
## 4 3/4/17    AQS 370030005   1                                0.046   ppm
## 5 3/5/17    AQS 370030005   1                                0.046   ppm
## 6 3/6/17    AQS 370030005   1                                0.048   ppm
##   DAILY_AQI_VALUE             Site.Name DAILY_OBS_COUNT PERCENT_COMPLETE
## 1              38 Taylorsville Liledoun              17              100
## 2              43 Taylorsville Liledoun              17              100
## 3              43 Taylorsville Liledoun              17              100
## 4              43 Taylorsville Liledoun              17              100
## 5              43 Taylorsville Liledoun              17              100
## 6              44 Taylorsville Liledoun              17              100
##   AQS_PARAMETER_CODE AQS_PARAMETER_DESC CBSA_CODE
## 1              44201              Ozone     25860
## 2              44201              Ozone     25860
## 3              44201              Ozone     25860
## 4              44201              Ozone     25860
## 5              44201              Ozone     25860
## 6              44201              Ozone     25860
##                      CBSA_NAME STATE_CODE          STATE COUNTY_CODE
## 1 Hickory-Lenoir-Morganton, NC         37 North Carolina           3
## 2 Hickory-Lenoir-Morganton, NC         37 North Carolina           3
## 3 Hickory-Lenoir-Morganton, NC         37 North Carolina           3
## 4 Hickory-Lenoir-Morganton, NC         37 North Carolina           3
## 5 Hickory-Lenoir-Morganton, NC         37 North Carolina           3
## 6 Hickory-Lenoir-Morganton, NC         37 North Carolina           3
##      COUNTY SITE_LATITUDE SITE_LONGITUDE
## 1 Alexander       35.9138        -81.191
## 2 Alexander       35.9138        -81.191
## 3 Alexander       35.9138        -81.191
## 4 Alexander       35.9138        -81.191
## 5 Alexander       35.9138        -81.191
## 6 Alexander       35.9138        -81.191
\end{verbatim}

\begin{Shaded}
\begin{Highlighting}[]
\KeywordTok{summary}\NormalTok{(NC.}\FloatTok{03.2017}\NormalTok{.raw.data}\OperatorTok{$}\NormalTok{Daily.Max.}\FloatTok{8.}\NormalTok{hour.Ozone.Concentration) }\CommentTok{#summary stats of 03 concentration}
\end{Highlighting}
\end{Shaded}

\begin{verbatim}
##    Min. 1st Qu.  Median    Mean 3rd Qu.    Max. 
## 0.00500 0.03500 0.04300 0.04211 0.04900 0.07500
\end{verbatim}

\begin{Shaded}
\begin{Highlighting}[]
\CommentTok{#Getting to know NC.03.2018 data}
\KeywordTok{dim}\NormalTok{(NC.}\FloatTok{03.2018}\NormalTok{.raw.data) }\CommentTok{#shows number of rows and columns in the dataset}
\end{Highlighting}
\end{Shaded}

\begin{verbatim}
## [1] 10781    20
\end{verbatim}

\begin{Shaded}
\begin{Highlighting}[]
\KeywordTok{str}\NormalTok{(NC.}\FloatTok{03.2018}\NormalTok{.raw.data) }\CommentTok{#shows the names and class of each variable and a sample of its values}
\end{Highlighting}
\end{Shaded}

\begin{verbatim}
## 'data.frame':    10781 obs. of  20 variables:
##  $ Date                                : Factor w/ 343 levels "1/1/18","1/10/18",..: 109 110 111 112 114 115 116 117 118 119 ...
##  $ Source                              : Factor w/ 2 levels "AirNow","AQS": 1 1 1 1 1 1 1 1 1 1 ...
##  $ Site.ID                             : int  370030005 370030005 370030005 370030005 370030005 370030005 370030005 370030005 370030005 370030005 ...
##  $ POC                                 : int  1 1 1 1 1 1 1 1 1 1 ...
##  $ Daily.Max.8.hour.Ozone.Concentration: num  0.038 0.033 0.04 0.02 0.019 0.021 0.031 0.022 0.038 0.031 ...
##  $ UNITS                               : Factor w/ 1 level "ppm": 1 1 1 1 1 1 1 1 1 1 ...
##  $ DAILY_AQI_VALUE                     : int  35 31 37 19 18 19 29 20 35 29 ...
##  $ Site.Name                           : Factor w/ 39 levels "","Beaufort",..: 34 34 34 34 34 34 34 34 34 34 ...
##  $ DAILY_OBS_COUNT                     : int  24 24 24 24 24 24 24 24 24 24 ...
##  $ PERCENT_COMPLETE                    : int  100 100 100 100 100 100 100 100 100 100 ...
##  $ AQS_PARAMETER_CODE                  : int  44201 44201 44201 44201 44201 44201 44201 44201 44201 44201 ...
##  $ AQS_PARAMETER_DESC                  : Factor w/ 1 level "Ozone": 1 1 1 1 1 1 1 1 1 1 ...
##  $ CBSA_CODE                           : int  25860 25860 25860 25860 25860 25860 25860 25860 25860 25860 ...
##  $ CBSA_NAME                           : Factor w/ 16 levels "","Asheville, NC",..: 8 8 8 8 8 8 8 8 8 8 ...
##  $ STATE_CODE                          : int  37 37 37 37 37 37 37 37 37 37 ...
##  $ STATE                               : Factor w/ 1 level "North Carolina": 1 1 1 1 1 1 1 1 1 1 ...
##  $ COUNTY_CODE                         : int  3 3 3 3 3 3 3 3 3 3 ...
##  $ COUNTY                              : Factor w/ 31 levels "Alexander","Avery",..: 1 1 1 1 1 1 1 1 1 1 ...
##  $ SITE_LATITUDE                       : num  35.9 35.9 35.9 35.9 35.9 ...
##  $ SITE_LONGITUDE                      : num  -81.2 -81.2 -81.2 -81.2 -81.2 ...
\end{verbatim}

\begin{Shaded}
\begin{Highlighting}[]
\KeywordTok{head}\NormalTok{(NC.}\FloatTok{03.2018}\NormalTok{.raw.data) }\CommentTok{#shows the first six observations in the dataset}
\end{Highlighting}
\end{Shaded}

\begin{verbatim}
##      Date Source   Site.ID POC Daily.Max.8.hour.Ozone.Concentration UNITS
## 1 2/16/18 AirNow 370030005   1                                0.038   ppm
## 2 2/17/18 AirNow 370030005   1                                0.033   ppm
## 3 2/18/18 AirNow 370030005   1                                0.040   ppm
## 4 2/19/18 AirNow 370030005   1                                0.020   ppm
## 5 2/20/18 AirNow 370030005   1                                0.019   ppm
## 6 2/21/18 AirNow 370030005   1                                0.021   ppm
##   DAILY_AQI_VALUE             Site.Name DAILY_OBS_COUNT PERCENT_COMPLETE
## 1              35 Taylorsville Liledoun              24              100
## 2              31 Taylorsville Liledoun              24              100
## 3              37 Taylorsville Liledoun              24              100
## 4              19 Taylorsville Liledoun              24              100
## 5              18 Taylorsville Liledoun              24              100
## 6              19 Taylorsville Liledoun              24              100
##   AQS_PARAMETER_CODE AQS_PARAMETER_DESC CBSA_CODE
## 1              44201              Ozone     25860
## 2              44201              Ozone     25860
## 3              44201              Ozone     25860
## 4              44201              Ozone     25860
## 5              44201              Ozone     25860
## 6              44201              Ozone     25860
##                      CBSA_NAME STATE_CODE          STATE COUNTY_CODE
## 1 Hickory-Lenoir-Morganton, NC         37 North Carolina           3
## 2 Hickory-Lenoir-Morganton, NC         37 North Carolina           3
## 3 Hickory-Lenoir-Morganton, NC         37 North Carolina           3
## 4 Hickory-Lenoir-Morganton, NC         37 North Carolina           3
## 5 Hickory-Lenoir-Morganton, NC         37 North Carolina           3
## 6 Hickory-Lenoir-Morganton, NC         37 North Carolina           3
##      COUNTY SITE_LATITUDE SITE_LONGITUDE
## 1 Alexander       35.9138        -81.191
## 2 Alexander       35.9138        -81.191
## 3 Alexander       35.9138        -81.191
## 4 Alexander       35.9138        -81.191
## 5 Alexander       35.9138        -81.191
## 6 Alexander       35.9138        -81.191
\end{verbatim}

\begin{Shaded}
\begin{Highlighting}[]
\CommentTok{#summary stats of daily 03 concentration}
\KeywordTok{summary}\NormalTok{(NC.}\FloatTok{03.2018}\NormalTok{.raw.data}\OperatorTok{$}\NormalTok{Daily.Max.}\FloatTok{8.}\NormalTok{hour.Ozone.Concentration) }
\end{Highlighting}
\end{Shaded}

\begin{verbatim}
##    Min. 1st Qu.  Median    Mean 3rd Qu.    Max. 
## 0.00000 0.03400 0.04100 0.04124 0.04900 0.07700
\end{verbatim}

\begin{Shaded}
\begin{Highlighting}[]
\CommentTok{#Getting to know NC.PM25.2017 data}
\KeywordTok{dim}\NormalTok{(NC.PM25.}\FloatTok{2017.}\NormalTok{raw.data) }\CommentTok{#shows number of rows and columns in the dataset}
\end{Highlighting}
\end{Shaded}

\begin{verbatim}
## [1] 9494   20
\end{verbatim}

\begin{Shaded}
\begin{Highlighting}[]
\KeywordTok{str}\NormalTok{(NC.PM25.}\FloatTok{2017.}\NormalTok{raw.data) }\CommentTok{#shows the names and class of each variable and a sample of its values}
\end{Highlighting}
\end{Shaded}

\begin{verbatim}
## 'data.frame':    9494 obs. of  20 variables:
##  $ Date                          : Factor w/ 365 levels "1/1/17","1/10/17",..: 1 26 29 2 5 8 11 15 18 21 ...
##  $ Source                        : Factor w/ 1 level "AQS": 1 1 1 1 1 1 1 1 1 1 ...
##  $ Site.ID                       : int  370110002 370110002 370110002 370110002 370110002 370110002 370110002 370110002 370110002 370110002 ...
##  $ POC                           : int  1 1 1 1 1 1 1 1 1 1 ...
##  $ Daily.Mean.PM2.5.Concentration: num  2.9 1.2 3.2 6.4 3.6 5.8 3.6 1.5 1.4 1.4 ...
##  $ UNITS                         : Factor w/ 1 level "ug/m3 LC": 1 1 1 1 1 1 1 1 1 1 ...
##  $ DAILY_AQI_VALUE               : int  12 5 13 27 15 24 15 6 6 6 ...
##  $ Site.Name                     : Factor w/ 25 levels "","Blackstone",..: 15 15 15 15 15 15 15 15 15 15 ...
##  $ DAILY_OBS_COUNT               : int  1 1 1 1 1 1 1 1 1 1 ...
##  $ PERCENT_COMPLETE              : int  100 100 100 100 100 100 100 100 100 100 ...
##  $ AQS_PARAMETER_CODE            : int  88502 88502 88502 88502 88502 88502 88502 88502 88502 88502 ...
##  $ AQS_PARAMETER_DESC            : Factor w/ 2 levels "Acceptable PM2.5 AQI & Speciation Mass",..: 1 1 1 1 1 1 1 1 1 1 ...
##  $ CBSA_CODE                     : int  NA NA NA NA NA NA NA NA NA NA ...
##  $ CBSA_NAME                     : Factor w/ 14 levels "","Asheville, NC",..: 1 1 1 1 1 1 1 1 1 1 ...
##  $ STATE_CODE                    : int  37 37 37 37 37 37 37 37 37 37 ...
##  $ STATE                         : Factor w/ 1 level "North Carolina": 1 1 1 1 1 1 1 1 1 1 ...
##  $ COUNTY_CODE                   : int  11 11 11 11 11 11 11 11 11 11 ...
##  $ COUNTY                        : Factor w/ 21 levels "Avery","Buncombe",..: 1 1 1 1 1 1 1 1 1 1 ...
##  $ SITE_LATITUDE                 : num  36 36 36 36 36 ...
##  $ SITE_LONGITUDE                : num  -81.9 -81.9 -81.9 -81.9 -81.9 ...
\end{verbatim}

\begin{Shaded}
\begin{Highlighting}[]
\KeywordTok{head}\NormalTok{(NC.PM25.}\FloatTok{2017.}\NormalTok{raw.data) }\CommentTok{#shows the first six observations in the dataset}
\end{Highlighting}
\end{Shaded}

\begin{verbatim}
##      Date Source   Site.ID POC Daily.Mean.PM2.5.Concentration    UNITS
## 1  1/1/17    AQS 370110002   1                            2.9 ug/m3 LC
## 2  1/4/17    AQS 370110002   1                            1.2 ug/m3 LC
## 3  1/7/17    AQS 370110002   1                            3.2 ug/m3 LC
## 4 1/10/17    AQS 370110002   1                            6.4 ug/m3 LC
## 5 1/13/17    AQS 370110002   1                            3.6 ug/m3 LC
## 6 1/16/17    AQS 370110002   1                            5.8 ug/m3 LC
##   DAILY_AQI_VALUE      Site.Name DAILY_OBS_COUNT PERCENT_COMPLETE
## 1              12 Linville Falls               1              100
## 2               5 Linville Falls               1              100
## 3              13 Linville Falls               1              100
## 4              27 Linville Falls               1              100
## 5              15 Linville Falls               1              100
## 6              24 Linville Falls               1              100
##   AQS_PARAMETER_CODE                     AQS_PARAMETER_DESC CBSA_CODE
## 1              88502 Acceptable PM2.5 AQI & Speciation Mass        NA
## 2              88502 Acceptable PM2.5 AQI & Speciation Mass        NA
## 3              88502 Acceptable PM2.5 AQI & Speciation Mass        NA
## 4              88502 Acceptable PM2.5 AQI & Speciation Mass        NA
## 5              88502 Acceptable PM2.5 AQI & Speciation Mass        NA
## 6              88502 Acceptable PM2.5 AQI & Speciation Mass        NA
##   CBSA_NAME STATE_CODE          STATE COUNTY_CODE COUNTY SITE_LATITUDE
## 1                   37 North Carolina          11  Avery      35.97235
## 2                   37 North Carolina          11  Avery      35.97235
## 3                   37 North Carolina          11  Avery      35.97235
## 4                   37 North Carolina          11  Avery      35.97235
## 5                   37 North Carolina          11  Avery      35.97235
## 6                   37 North Carolina          11  Avery      35.97235
##   SITE_LONGITUDE
## 1      -81.93307
## 2      -81.93307
## 3      -81.93307
## 4      -81.93307
## 5      -81.93307
## 6      -81.93307
\end{verbatim}

\begin{Shaded}
\begin{Highlighting}[]
\CommentTok{#summary stats of daily PM25 concentation}
\KeywordTok{summary}\NormalTok{(NC.PM25.}\FloatTok{2017.}\NormalTok{raw.data}\OperatorTok{$}\NormalTok{Daily.Mean.PM2.}\FloatTok{5.}\NormalTok{Concentration) }
\end{Highlighting}
\end{Shaded}

\begin{verbatim}
##    Min. 1st Qu.  Median    Mean 3rd Qu.    Max. 
##  -3.900   5.000   7.300   7.742  10.000  31.900
\end{verbatim}

\begin{Shaded}
\begin{Highlighting}[]
\CommentTok{#Getting to know NC.PM25.2018 data}
\KeywordTok{dim}\NormalTok{(NC.PM25.}\FloatTok{2018.}\NormalTok{raw.data) }\CommentTok{#shows number of rows and columns in the dataset}
\end{Highlighting}
\end{Shaded}

\begin{verbatim}
## [1] 7611   20
\end{verbatim}

\begin{Shaded}
\begin{Highlighting}[]
\KeywordTok{str}\NormalTok{(NC.PM25.}\FloatTok{2018.}\NormalTok{raw.data) }\CommentTok{#shows the names and class of each variable and a sample of its values}
\end{Highlighting}
\end{Shaded}

\begin{verbatim}
## 'data.frame':    7611 obs. of  20 variables:
##  $ Date                          : Factor w/ 343 levels "1/1/18","1/10/18",..: 12 27 30 3 6 9 13 16 19 22 ...
##  $ Source                        : Factor w/ 2 levels "AirNow","AQS": 2 2 2 2 2 2 2 2 2 2 ...
##  $ Site.ID                       : int  370110002 370110002 370110002 370110002 370110002 370110002 370110002 370110002 370110002 370110002 ...
##  $ POC                           : int  1 1 1 1 1 1 1 1 1 1 ...
##  $ Daily.Mean.PM2.5.Concentration: num  2.9 3.7 5.3 0.8 2.5 4.5 1.8 2.5 4.2 1.7 ...
##  $ UNITS                         : Factor w/ 1 level "ug/m3 LC": 1 1 1 1 1 1 1 1 1 1 ...
##  $ DAILY_AQI_VALUE               : int  12 15 22 3 10 19 8 10 18 7 ...
##  $ Site.Name                     : Factor w/ 24 levels "","Blackstone",..: 14 14 14 14 14 14 14 14 14 14 ...
##  $ DAILY_OBS_COUNT               : int  1 1 1 1 1 1 1 1 1 1 ...
##  $ PERCENT_COMPLETE              : int  100 100 100 100 100 100 100 100 100 100 ...
##  $ AQS_PARAMETER_CODE            : int  88502 88502 88502 88502 88502 88502 88502 88502 88502 88502 ...
##  $ AQS_PARAMETER_DESC            : Factor w/ 2 levels "Acceptable PM2.5 AQI & Speciation Mass",..: 1 1 1 1 1 1 1 1 1 1 ...
##  $ CBSA_CODE                     : int  NA NA NA NA NA NA NA NA NA NA ...
##  $ CBSA_NAME                     : Factor w/ 14 levels "","Asheville, NC",..: 1 1 1 1 1 1 1 1 1 1 ...
##  $ STATE_CODE                    : int  37 37 37 37 37 37 37 37 37 37 ...
##  $ STATE                         : Factor w/ 1 level "North Carolina": 1 1 1 1 1 1 1 1 1 1 ...
##  $ COUNTY_CODE                   : int  11 11 11 11 11 11 11 11 11 11 ...
##  $ COUNTY                        : Factor w/ 21 levels "Avery","Buncombe",..: 1 1 1 1 1 1 1 1 1 1 ...
##  $ SITE_LATITUDE                 : num  36 36 36 36 36 ...
##  $ SITE_LONGITUDE                : num  -81.9 -81.9 -81.9 -81.9 -81.9 ...
\end{verbatim}

\begin{Shaded}
\begin{Highlighting}[]
\KeywordTok{head}\NormalTok{(NC.PM25.}\FloatTok{2018.}\NormalTok{raw.data) }\CommentTok{#shows the first six observations in the dataset}
\end{Highlighting}
\end{Shaded}

\begin{verbatim}
##      Date Source   Site.ID POC Daily.Mean.PM2.5.Concentration    UNITS
## 1  1/2/18    AQS 370110002   1                            2.9 ug/m3 LC
## 2  1/5/18    AQS 370110002   1                            3.7 ug/m3 LC
## 3  1/8/18    AQS 370110002   1                            5.3 ug/m3 LC
## 4 1/11/18    AQS 370110002   1                            0.8 ug/m3 LC
## 5 1/14/18    AQS 370110002   1                            2.5 ug/m3 LC
## 6 1/17/18    AQS 370110002   1                            4.5 ug/m3 LC
##   DAILY_AQI_VALUE      Site.Name DAILY_OBS_COUNT PERCENT_COMPLETE
## 1              12 Linville Falls               1              100
## 2              15 Linville Falls               1              100
## 3              22 Linville Falls               1              100
## 4               3 Linville Falls               1              100
## 5              10 Linville Falls               1              100
## 6              19 Linville Falls               1              100
##   AQS_PARAMETER_CODE                     AQS_PARAMETER_DESC CBSA_CODE
## 1              88502 Acceptable PM2.5 AQI & Speciation Mass        NA
## 2              88502 Acceptable PM2.5 AQI & Speciation Mass        NA
## 3              88502 Acceptable PM2.5 AQI & Speciation Mass        NA
## 4              88502 Acceptable PM2.5 AQI & Speciation Mass        NA
## 5              88502 Acceptable PM2.5 AQI & Speciation Mass        NA
## 6              88502 Acceptable PM2.5 AQI & Speciation Mass        NA
##   CBSA_NAME STATE_CODE          STATE COUNTY_CODE COUNTY SITE_LATITUDE
## 1                   37 North Carolina          11  Avery      35.97235
## 2                   37 North Carolina          11  Avery      35.97235
## 3                   37 North Carolina          11  Avery      35.97235
## 4                   37 North Carolina          11  Avery      35.97235
## 5                   37 North Carolina          11  Avery      35.97235
## 6                   37 North Carolina          11  Avery      35.97235
##   SITE_LONGITUDE
## 1      -81.93307
## 2      -81.93307
## 3      -81.93307
## 4      -81.93307
## 5      -81.93307
## 6      -81.93307
\end{verbatim}

\begin{Shaded}
\begin{Highlighting}[]
\CommentTok{#summary stats of daily PM25 concentation}
\KeywordTok{summary}\NormalTok{(NC.PM25.}\FloatTok{2018.}\NormalTok{raw.data}\OperatorTok{$}\NormalTok{Daily.Mean.PM2.}\FloatTok{5.}\NormalTok{Concentration) }
\end{Highlighting}
\end{Shaded}

\begin{verbatim}
##    Min. 1st Qu.  Median    Mean 3rd Qu.    Max. 
##  -2.800   5.000   7.200   7.554   9.800  34.200
\end{verbatim}

\subsection{Wrangle individual datasets to create processed
files.}\label{wrangle-individual-datasets-to-create-processed-files.}

\begin{enumerate}
\def\labelenumi{\arabic{enumi}.}
\setcounter{enumi}{2}
\tightlist
\item
  Change date to date
\item
  Select the following columns: Date, DAILY\_AQI\_VALUE, Site.Name,
  AQS\_PARAMETER\_DESC, COUNTY, SITE\_LATITUDE, SITE\_LONGITUDE
\item
  For the PM2.5 datasets, fill all cells in AQS\_PARAMETER\_DESC with
  ``PM2.5'' (all cells in this column should be identical).
\item
  Save all four processed datasets in the Processed folder.
\end{enumerate}

\begin{Shaded}
\begin{Highlighting}[]
\CommentTok{#3}
\CommentTok{#Changing date variable of NC.03.2017 data to date format}
\NormalTok{NC.}\FloatTok{03.2017}\NormalTok{.raw.data}\OperatorTok{$}\NormalTok{Date <-}\StringTok{ }\KeywordTok{as.Date}\NormalTok{(NC.}\FloatTok{03.2017}\NormalTok{.raw.data}\OperatorTok{$}\NormalTok{Date, }\DataTypeTok{format =} \StringTok{"%m/%d/%y"}\NormalTok{)}
\KeywordTok{class}\NormalTok{(NC.}\FloatTok{03.2017}\NormalTok{.raw.data}\OperatorTok{$}\NormalTok{Date) }\CommentTok{#confirming date change}
\end{Highlighting}
\end{Shaded}

\begin{verbatim}
## [1] "Date"
\end{verbatim}

\begin{Shaded}
\begin{Highlighting}[]
\CommentTok{#Changing date variable of NC.03.2018 data to date format}
\NormalTok{NC.}\FloatTok{03.2018}\NormalTok{.raw.data}\OperatorTok{$}\NormalTok{Date <-}\StringTok{ }\KeywordTok{as.Date}\NormalTok{(NC.}\FloatTok{03.2018}\NormalTok{.raw.data}\OperatorTok{$}\NormalTok{Date, }\DataTypeTok{format =} \StringTok{"%m/%d/%y"}\NormalTok{)}
\KeywordTok{class}\NormalTok{(NC.}\FloatTok{03.2018}\NormalTok{.raw.data}\OperatorTok{$}\NormalTok{Date) }\CommentTok{#confirming date change}
\end{Highlighting}
\end{Shaded}

\begin{verbatim}
## [1] "Date"
\end{verbatim}

\begin{Shaded}
\begin{Highlighting}[]
\CommentTok{#Changing date variable of NC.PM25.2017 data to date format}
\NormalTok{NC.PM25.}\FloatTok{2017.}\NormalTok{raw.data}\OperatorTok{$}\NormalTok{Date <-}\StringTok{ }\KeywordTok{as.Date}\NormalTok{(NC.PM25.}\FloatTok{2017.}\NormalTok{raw.data}\OperatorTok{$}\NormalTok{Date, }\DataTypeTok{format =} \StringTok{"%m/%d/%y"}\NormalTok{)}
\KeywordTok{class}\NormalTok{(NC.PM25.}\FloatTok{2017.}\NormalTok{raw.data}\OperatorTok{$}\NormalTok{Date) }\CommentTok{#confirming date change}
\end{Highlighting}
\end{Shaded}

\begin{verbatim}
## [1] "Date"
\end{verbatim}

\begin{Shaded}
\begin{Highlighting}[]
\CommentTok{#Changing date variable of NC.PM25.2018 data to date format}
\NormalTok{NC.PM25.}\FloatTok{2018.}\NormalTok{raw.data}\OperatorTok{$}\NormalTok{Date <-}\StringTok{ }\KeywordTok{as.Date}\NormalTok{(NC.PM25.}\FloatTok{2018.}\NormalTok{raw.data}\OperatorTok{$}\NormalTok{Date, }\DataTypeTok{format =} \StringTok{"%m/%d/%y"}\NormalTok{)}
\KeywordTok{class}\NormalTok{(NC.PM25.}\FloatTok{2018.}\NormalTok{raw.data}\OperatorTok{$}\NormalTok{Date) }\CommentTok{#confirming date change}
\end{Highlighting}
\end{Shaded}

\begin{verbatim}
## [1] "Date"
\end{verbatim}

\begin{Shaded}
\begin{Highlighting}[]
\CommentTok{#4}
\CommentTok{#selecting specific columns in the NC.03.2017 data}
\NormalTok{NC.}\FloatTok{03.2017}\NormalTok{.proccessed.v1 <-}\StringTok{ }\KeywordTok{select}\NormalTok{(NC.}\FloatTok{03.2017}\NormalTok{.raw.data, }\StringTok{"Date"}\NormalTok{, }\StringTok{"DAILY_AQI_VALUE"}\NormalTok{, }
                                   \StringTok{"Site.Name"}\NormalTok{, }\StringTok{"AQS_PARAMETER_DESC"}\NormalTok{, }
                                   \StringTok{"COUNTY"}\NormalTok{, }\StringTok{"SITE_LATITUDE"}\NormalTok{, }\StringTok{"SITE_LONGITUDE"}\NormalTok{)}

\CommentTok{#selecting specific columns in the NC.03.2018 data}
\NormalTok{NC.}\FloatTok{03.2018}\NormalTok{.proccessed.v1 <-}\StringTok{ }\KeywordTok{select}\NormalTok{(NC.}\FloatTok{03.2018}\NormalTok{.raw.data, }\StringTok{"Date"}\NormalTok{, }\StringTok{"DAILY_AQI_VALUE"}\NormalTok{, }
                                   \StringTok{"Site.Name"}\NormalTok{, }\StringTok{"AQS_PARAMETER_DESC"}\NormalTok{, }
                                   \StringTok{"COUNTY"}\NormalTok{, }\StringTok{"SITE_LATITUDE"}\NormalTok{, }\StringTok{"SITE_LONGITUDE"}\NormalTok{)}

\CommentTok{#selecting specific columns in the NC.PM25.2017 data}
\NormalTok{NC.PM25.}\FloatTok{2017.}\NormalTok{proccessed.v1 <-}\StringTok{ }\KeywordTok{select}\NormalTok{(NC.PM25.}\FloatTok{2017.}\NormalTok{raw.data, }\StringTok{"Date"}\NormalTok{, }\StringTok{"DAILY_AQI_VALUE"}\NormalTok{, }
                                     \StringTok{"Site.Name"}\NormalTok{, }\StringTok{"AQS_PARAMETER_DESC"}\NormalTok{, }
                                     \StringTok{"COUNTY"}\NormalTok{, }\StringTok{"SITE_LATITUDE"}\NormalTok{, }\StringTok{"SITE_LONGITUDE"}\NormalTok{)}

\CommentTok{#selecting specific columns in the NC.PM25.2018 data}
\NormalTok{NC.PM25.}\FloatTok{2018.}\NormalTok{proccessed.v1 <-}\StringTok{ }\KeywordTok{select}\NormalTok{(NC.PM25.}\FloatTok{2018.}\NormalTok{raw.data, }\StringTok{"Date"}\NormalTok{, }\StringTok{"DAILY_AQI_VALUE"}\NormalTok{, }
                                     \StringTok{"Site.Name"}\NormalTok{, }\StringTok{"AQS_PARAMETER_DESC"}\NormalTok{, }\StringTok{"COUNTY"}\NormalTok{, }
                                     \StringTok{"SITE_LATITUDE"}\NormalTok{, }\StringTok{"SITE_LONGITUDE"}\NormalTok{)}

\CommentTok{#5}
\CommentTok{#filling all cells in dataset NC.PM25.2017.proccessed.v1, variable AQS_PARAMETER_DESC with "PM2.5"}
\NormalTok{NC.PM25.}\FloatTok{2017.}\NormalTok{proccessed.v2 <-}\StringTok{ }\KeywordTok{mutate}\NormalTok{(NC.PM25.}\FloatTok{2017.}\NormalTok{proccessed.v1,}\DataTypeTok{AQS_PARAMETER_DESC =} \StringTok{"PM2.5"}\NormalTok{)}

\CommentTok{#filling all cells in dataset NC.PM25.2018.proccessed.v1, variable AQS_PARAMETER_DESC with "PM2.5"}
\NormalTok{NC.PM25.}\FloatTok{2018.}\NormalTok{proccessed.v2 <-}\StringTok{ }\KeywordTok{mutate}\NormalTok{(NC.PM25.}\FloatTok{2018.}\NormalTok{proccessed.v1,}\DataTypeTok{AQS_PARAMETER_DESC =} \StringTok{"PM2.5"}\NormalTok{)}

\CommentTok{#6}
\CommentTok{#Saving NC.03.2017.proccessed.v1 in processed data folder}
\KeywordTok{write.csv}\NormalTok{(NC.}\FloatTok{03.2017}\NormalTok{.proccessed.v1, }\DataTypeTok{row.names =} \OtherTok{FALSE}\NormalTok{, }\DataTypeTok{file =} \StringTok{"./Data/Processed/NC.03.2017.proccessed.v1.csv"}\NormalTok{)}

\CommentTok{#Saving NC.03.2018.proccessed.v1 in processed data folder}
\KeywordTok{write.csv}\NormalTok{(NC.}\FloatTok{03.2018}\NormalTok{.proccessed.v1, }\DataTypeTok{row.names =} \OtherTok{FALSE}\NormalTok{, }\DataTypeTok{file =} \StringTok{"./Data/Processed/NC.03.2018.proccessed.v1.csv"}\NormalTok{)}

\CommentTok{#Saving NC.PM25.2017.proccessed.v2 in processed data folder}
\KeywordTok{write.csv}\NormalTok{(NC.PM25.}\FloatTok{2017.}\NormalTok{proccessed.v2, }\DataTypeTok{row.names =} \OtherTok{FALSE}\NormalTok{, }\DataTypeTok{file =} \StringTok{"./Data/Processed/NC.PM25.2017.proccessed.v2.csv"}\NormalTok{)}

\CommentTok{#Saving NC.PM25.2018.proccessed.v2 in processed data folder}
\KeywordTok{write.csv}\NormalTok{(NC.PM25.}\FloatTok{2018.}\NormalTok{proccessed.v2, }\DataTypeTok{row.names =} \OtherTok{FALSE}\NormalTok{, }\DataTypeTok{file =} \StringTok{"./Data/Processed/NC.PM25.2018.proccessed.v2.csv"}\NormalTok{)}
\end{Highlighting}
\end{Shaded}

\subsection{Combine datasets}\label{combine-datasets}

\begin{enumerate}
\def\labelenumi{\arabic{enumi}.}
\setcounter{enumi}{6}
\tightlist
\item
  Combine the four datasets with \texttt{rbind}. Make sure your column
  names are identical prior to running this code.
\item
  Wrangle your new dataset with a pipe function (\%\textgreater{}\%) so
  that it fills the following conditions:
\end{enumerate}

\begin{itemize}
\tightlist
\item
  Sites: Blackstone, Bryson City, Triple Oak
\item
  Add columns for ``Month'' and ``Year'' by parsing your ``Date'' column
  (hint: \texttt{separate} function or \texttt{lubridate} package)
\end{itemize}

\begin{enumerate}
\def\labelenumi{\arabic{enumi}.}
\setcounter{enumi}{8}
\tightlist
\item
  Spread your datasets such that AQI values for ozone and PM2.5 are in
  separate columns. Each location on a specific date should now occupy
  only one row.
\item
  Call up the dimensions of your new tidy dataset.
\item
  Save your processed dataset with the following file name:
  ``EPAair\_O3\_PM25\_NC1718\_Processed.csv''
\end{enumerate}

\begin{Shaded}
\begin{Highlighting}[]
\CommentTok{#7}
\CommentTok{#Ensuring all column names are identical}
\KeywordTok{colnames}\NormalTok{(NC.}\FloatTok{03.2017}\NormalTok{.proccessed.v1)}
\end{Highlighting}
\end{Shaded}

\begin{verbatim}
## [1] "Date"               "DAILY_AQI_VALUE"    "Site.Name"         
## [4] "AQS_PARAMETER_DESC" "COUNTY"             "SITE_LATITUDE"     
## [7] "SITE_LONGITUDE"
\end{verbatim}

\begin{Shaded}
\begin{Highlighting}[]
\KeywordTok{colnames}\NormalTok{(NC.}\FloatTok{03.2018}\NormalTok{.proccessed.v1)}
\end{Highlighting}
\end{Shaded}

\begin{verbatim}
## [1] "Date"               "DAILY_AQI_VALUE"    "Site.Name"         
## [4] "AQS_PARAMETER_DESC" "COUNTY"             "SITE_LATITUDE"     
## [7] "SITE_LONGITUDE"
\end{verbatim}

\begin{Shaded}
\begin{Highlighting}[]
\KeywordTok{colnames}\NormalTok{(NC.PM25.}\FloatTok{2017.}\NormalTok{proccessed.v2)}
\end{Highlighting}
\end{Shaded}

\begin{verbatim}
## [1] "Date"               "DAILY_AQI_VALUE"    "Site.Name"         
## [4] "AQS_PARAMETER_DESC" "COUNTY"             "SITE_LATITUDE"     
## [7] "SITE_LONGITUDE"
\end{verbatim}

\begin{Shaded}
\begin{Highlighting}[]
\KeywordTok{colnames}\NormalTok{(NC.PM25.}\FloatTok{2018.}\NormalTok{proccessed.v2)}
\end{Highlighting}
\end{Shaded}

\begin{verbatim}
## [1] "Date"               "DAILY_AQI_VALUE"    "Site.Name"         
## [4] "AQS_PARAMETER_DESC" "COUNTY"             "SITE_LATITUDE"     
## [7] "SITE_LONGITUDE"
\end{verbatim}

\begin{Shaded}
\begin{Highlighting}[]
\CommentTok{#Combining all datasets using rbind}
\NormalTok{NC.}\FloatTok{03.}\NormalTok{PM25.}\FloatTok{2017.2018}\NormalTok{.data <-}\StringTok{ }\KeywordTok{rbind}\NormalTok{(NC.}\FloatTok{03.2017}\NormalTok{.proccessed.v1,NC.}\FloatTok{03.2018}\NormalTok{.proccessed.v1,NC.PM25.}\FloatTok{2017.}\NormalTok{proccessed.v2,NC.PM25.}\FloatTok{2018.}\NormalTok{proccessed.v2)}

\CommentTok{#8 #Wrangling dataset}
\CommentTok{#displaying the different factor levels of Site.name}
\KeywordTok{levels}\NormalTok{(NC.}\FloatTok{03.}\NormalTok{PM25.}\FloatTok{2017.2018}\NormalTok{.data}\OperatorTok{$}\NormalTok{Site.Name) }
\end{Highlighting}
\end{Shaded}

\begin{verbatim}
##  [1] ""                                                                
##  [2] "Beaufort"                                                        
##  [3] "Bent Creek"                                                      
##  [4] "Bethany sch."                                                    
##  [5] "Blackstone"                                                      
##  [6] "Bryson City"                                                     
##  [7] "Bushy Fork"                                                      
##  [8] "Butner"                                                          
##  [9] "Candor"                                                          
## [10] "Castle Hayne"                                                    
## [11] "Cherry Grove"                                                    
## [12] "Clemmons Middle"                                                 
## [13] "Coweeta"                                                         
## [14] "Cranberry"                                                       
## [15] "Crouse"                                                          
## [16] "Durham Armory"                                                   
## [17] "Frying Pan Mountain"                                             
## [18] "Garinger High School"                                            
## [19] "Hattie Avenue"                                                   
## [20] "Honeycutt School"                                                
## [21] "Jamesville School"                                               
## [22] "Joanna Bald"                                                     
## [23] "Leggett"                                                         
## [24] "Lenoir (city)"                                                   
## [25] "Lenoir Co. Comm. Coll."                                          
## [26] "Linville Falls"                                                  
## [27] "Mendenhall School"                                               
## [28] "Millbrook School"                                                
## [29] "Monroe School"                                                   
## [30] "Mt. Mitchell"                                                    
## [31] "OZONE MONITOR ON SW SIDE OF TOWER/MET EQUIPMENT 10FT ABOVE TOWER"
## [32] "Pitt Agri. Center"                                               
## [33] "Purchase Knob"                                                   
## [34] "Rockwell"                                                        
## [35] "Taylorsville Liledoun"                                           
## [36] "Union Cross"                                                     
## [37] "University Meadows"                                              
## [38] "Wade"                                                            
## [39] "Waynesville School"                                              
## [40] "West Johnston Co."                                               
## [41] "Board Of Ed. Bldg."                                              
## [42] "Candor: EPA CASTNet Site"                                        
## [43] "Hickory Water Tower"                                             
## [44] "Lexington water tower"                                           
## [45] "Montclaire Elementary School"                                    
## [46] "PM2.5 COLOCATED MONITORS LOCATED ON TOP OF BUILDING"             
## [47] "Remount"                                                         
## [48] "Spruce Pine Hospital"                                            
## [49] "Triple Oak"                                                      
## [50] "William Owen School"
\end{verbatim}

\begin{Shaded}
\begin{Highlighting}[]
\NormalTok{NC.}\FloatTok{03.}\NormalTok{PM25.}\FloatTok{2017.2018}\NormalTok{.data.v1 <-}\StringTok{ }\NormalTok{NC.}\FloatTok{03.}\NormalTok{PM25.}\FloatTok{2017.2018}\NormalTok{.data }\OperatorTok
\StringTok{  }\CommentTok{#filtering out data from sites Blackstone, Bryson City, Triple Oak}
\StringTok{  }\KeywordTok{filter}\NormalTok{(Site.Name}\OperatorTok{==}\StringTok{"Blackstone"}\OperatorTok{|}\NormalTok{Site.Name}\OperatorTok{==}\StringTok{"Bryson City"}\OperatorTok{|}\NormalTok{Site.Name}\OperatorTok{==}\StringTok{"Triple Oak"}\NormalTok{) }\OperatorTok\StringTok{  }
\StringTok{  }\KeywordTok{mutate}\NormalTok{(}\DataTypeTok{Month =} \KeywordTok{month}\NormalTok{(Date)) }\OperatorTok\StringTok{ }\CommentTok{#including a month column}
\StringTok{  }\KeywordTok{mutate}\NormalTok{(}\DataTypeTok{Year =} \KeywordTok{year}\NormalTok{(Date)) }\CommentTok{#including a year column}
 

\CommentTok{#9}
\CommentTok{#spreading dataset to include 2 columns for DAILY_AQI_VALUEs,broken down by AQS_PARAMETER_DESC factors Ozone and PM 2.5.}
\NormalTok{NC.}\FloatTok{03.}\NormalTok{PM25.}\FloatTok{2017.2018}\NormalTok{.data.v2 <-}\StringTok{ }\NormalTok{NC.}\FloatTok{03.}\NormalTok{PM25.}\FloatTok{2017.2018}\NormalTok{.data.v1 }\OperatorTok
\StringTok{  }\KeywordTok{spread}\NormalTok{(AQS_PARAMETER_DESC,DAILY_AQI_VALUE) }\OperatorTok\StringTok{ }
\StringTok{  }\KeywordTok{rename}\NormalTok{(}\DataTypeTok{Ozone_Daily_AQI=}\NormalTok{Ozone,}\DataTypeTok{PM2.5_Daily_AQI=}\NormalTok{PM2.}\DecValTok{5}\NormalTok{) }\CommentTok{#renaming columns to more descriptive data label}

\CommentTok{#10}
\CommentTok{#Dimensions of the new dataset}
\KeywordTok{dim}\NormalTok{(NC.}\FloatTok{03.}\NormalTok{PM25.}\FloatTok{2017.2018}\NormalTok{.data.v2)}
\end{Highlighting}
\end{Shaded}

\begin{verbatim}
## [1] 1953    9
\end{verbatim}

\begin{Shaded}
\begin{Highlighting}[]
\CommentTok{#11}
\CommentTok{#saving the dataset in the processed folder}
\KeywordTok{write.csv}\NormalTok{(NC.}\FloatTok{03.}\NormalTok{PM25.}\FloatTok{2017.2018}\NormalTok{.data.v2, }\DataTypeTok{row.names =} \OtherTok{FALSE}\NormalTok{, }\DataTypeTok{file =} \StringTok{"./Data/Processed/EPAair_O3_PM25_NC1718_Processed.csv"}\NormalTok{)}
\end{Highlighting}
\end{Shaded}

\subsection{Generate summary tables}\label{generate-summary-tables}

\begin{enumerate}
\def\labelenumi{\arabic{enumi}.}
\setcounter{enumi}{11}
\tightlist
\item
  Use the split-apply-combine strategy to generate two new data frames:
\end{enumerate}

\begin{enumerate}
\def\labelenumi{\alph{enumi}.}
\tightlist
\item
  A summary table of mean AQI values for O3 and PM2.5 by month
\item
  A summary table of the mean, minimum, and maximum AQI values of O3 and
  PM2.5 for each site
\end{enumerate}

\begin{enumerate}
\def\labelenumi{\arabic{enumi}.}
\setcounter{enumi}{12}
\tightlist
\item
  Display the data frames.
\end{enumerate}

\begin{Shaded}
\begin{Highlighting}[]
\CommentTok{#12a }
\CommentTok{#summary table of mean AQI values for O3 and PM2.5 by month}
\NormalTok{NC.}\FloatTok{03.}\NormalTok{PM25.}\FloatTok{2017.2018}\NormalTok{.data.month.summ <-}\StringTok{ }
\StringTok{  }\NormalTok{NC.}\FloatTok{03.}\NormalTok{PM25.}\FloatTok{2017.2018}\NormalTok{.data.v2 }\OperatorTok
\StringTok{  }\KeywordTok{group_by}\NormalTok{(Month) }\OperatorTok
\StringTok{  }\KeywordTok{summarise}\NormalTok{(}\DataTypeTok{mean.AQI.O3 =} \KeywordTok{mean}\NormalTok{(Ozone_Daily_AQI,}\DataTypeTok{na.rm=}\OtherTok{TRUE}\NormalTok{),}
            \DataTypeTok{mean.AQI.PM2.5 =} \KeywordTok{mean}\NormalTok{(PM2.5_Daily_AQI,}\DataTypeTok{na.rm=}\OtherTok{TRUE}\NormalTok{))}
            \CommentTok{#na.rm=TRUE excludes NA values in the mean computation}
          
\CommentTok{#12b}
\CommentTok{#summary table of mean, minimum, and maximum AQI values of O3 and PM2.5 for each site}
\NormalTok{NC.}\FloatTok{03.}\NormalTok{PM25.}\FloatTok{2017.2018}\NormalTok{.data.site.summ <-}\StringTok{ }
\StringTok{  }\NormalTok{NC.}\FloatTok{03.}\NormalTok{PM25.}\FloatTok{2017.2018}\NormalTok{.data.v2 }\OperatorTok
\StringTok{  }\KeywordTok{group_by}\NormalTok{(Site.Name) }\OperatorTok
\StringTok{  }\KeywordTok{summarise}\NormalTok{(}\DataTypeTok{mean.AQI.O3 =} \KeywordTok{mean}\NormalTok{(Ozone_Daily_AQI,}\DataTypeTok{na.rm=}\OtherTok{TRUE}\NormalTok{),}
            \DataTypeTok{mean.AQI.PM2.5 =} \KeywordTok{mean}\NormalTok{(PM2.5_Daily_AQI,}\DataTypeTok{na.rm=}\OtherTok{TRUE}\NormalTok{),}
            \DataTypeTok{min.AQI.O3 =} \KeywordTok{min}\NormalTok{(Ozone_Daily_AQI,}\DataTypeTok{na.rm=}\OtherTok{TRUE}\NormalTok{),}
            \DataTypeTok{min.AQI.PM2.5 =} \KeywordTok{min}\NormalTok{(PM2.5_Daily_AQI,}\DataTypeTok{na.rm=}\OtherTok{TRUE}\NormalTok{),}
             \DataTypeTok{max.AQI.O3 =} \KeywordTok{max}\NormalTok{(Ozone_Daily_AQI,}\DataTypeTok{na.rm=}\OtherTok{TRUE}\NormalTok{),}
            \DataTypeTok{max.AQI.PM2.5 =} \KeywordTok{max}\NormalTok{(PM2.5_Daily_AQI,}\DataTypeTok{na.rm=}\OtherTok{TRUE}\NormalTok{))}
           \CommentTok{#na.rm=TRUE excludes NA values in the mean computation}

\CommentTok{#13}
\CommentTok{#Displaying the summary table of mean AQI values for O3 and PM2.5 by month}
\KeywordTok{kable}\NormalTok{(NC.}\FloatTok{03.}\NormalTok{PM25.}\FloatTok{2017.2018}\NormalTok{.data.month.summ, }\DataTypeTok{caption =} \StringTok{"Summary table of mean AQI values by month"}\NormalTok{)}
\end{Highlighting}
\end{Shaded}

\begin{longtable}[]{@{}rrr@{}}
\caption{Summary table of mean AQI values by month}\tabularnewline
\toprule
Month & mean.AQI.O3 & mean.AQI.PM2.5\tabularnewline
\midrule
\endfirsthead
\toprule
Month & mean.AQI.O3 & mean.AQI.PM2.5\tabularnewline
\midrule
\endhead
1 & 31.48276 & 34.58192\tabularnewline
2 & 35.52174 & 36.70659\tabularnewline
3 & 42.40164 & 35.13978\tabularnewline
4 & 44.30000 & 32.52147\tabularnewline
5 & 38.90826 & 31.68333\tabularnewline
6 & 38.71429 & 33.28743\tabularnewline
7 & 38.16129 & 33.07609\tabularnewline
8 & 33.95960 & 33.68667\tabularnewline
9 & 32.59036 & 31.88889\tabularnewline
10 & 32.12644 & 29.32639\tabularnewline
11 & 30.06897 & 36.83333\tabularnewline
12 & 29.78378 & 41.12150\tabularnewline
\bottomrule
\end{longtable}

\begin{Shaded}
\begin{Highlighting}[]
\CommentTok{#Displaying the summary table of mean, minimum, and maximum AQI values of O3 and PM2.5 for each site}
\KeywordTok{kable}\NormalTok{(NC.}\FloatTok{03.}\NormalTok{PM25.}\FloatTok{2017.2018}\NormalTok{.data.site.summ, }\DataTypeTok{caption =} \StringTok{"Summary table of mean,min and max AQI values by site"}\NormalTok{)}
\end{Highlighting}
\end{Shaded}

\begin{longtable}[]{@{}lrrrrrr@{}}
\caption{Summary table of mean,min and max AQI values by
site}\tabularnewline
\toprule
Site.Name & mean.AQI.O3 & mean.AQI.PM2.5 & min.AQI.O3 & min.AQI.PM2.5 &
max.AQI.O3 & max.AQI.PM2.5\tabularnewline
\midrule
\endfirsthead
\toprule
Site.Name & mean.AQI.O3 & mean.AQI.PM2.5 & min.AQI.O3 & min.AQI.PM2.5 &
max.AQI.O3 & max.AQI.PM2.5\tabularnewline
\midrule
\endhead
Blackstone & 38.48246 & 36.72613 & 8 & 0 & 97 & 83\tabularnewline
Bryson City & 35.18252 & 32.29955 & 5 & 3 & 71 & 78\tabularnewline
Triple Oak & NaN & 33.48000 & Inf & 0 & -Inf & 74\tabularnewline
\bottomrule
\end{longtable}


\end{document}
