\documentclass[]{article}
\usepackage{lmodern}
\usepackage{amssymb,amsmath}
\usepackage{ifxetex,ifluatex}
\usepackage{fixltx2e} % provides \textsubscript
\ifnum 0\ifxetex 1\fi\ifluatex 1\fi=0 % if pdftex
  \usepackage[T1]{fontenc}
  \usepackage[utf8]{inputenc}
\else % if luatex or xelatex
  \ifxetex
    \usepackage{mathspec}
  \else
    \usepackage{fontspec}
  \fi
  \defaultfontfeatures{Ligatures=TeX,Scale=MatchLowercase}
\fi
% use upquote if available, for straight quotes in verbatim environments
\IfFileExists{upquote.sty}{\usepackage{upquote}}{}
% use microtype if available
\IfFileExists{microtype.sty}{%
\usepackage{microtype}
\UseMicrotypeSet[protrusion]{basicmath} % disable protrusion for tt fonts
}{}
\usepackage[margin=2.54cm]{geometry}
\usepackage{hyperref}
\hypersetup{unicode=true,
            pdftitle={Assignment 3: Data Exploration},
            pdfauthor={Njeri Kara},
            pdfborder={0 0 0},
            breaklinks=true}
\urlstyle{same}  % don't use monospace font for urls
\usepackage{color}
\usepackage{fancyvrb}
\newcommand{\VerbBar}{|}
\newcommand{\VERB}{\Verb[commandchars=\\\{\}]}
\DefineVerbatimEnvironment{Highlighting}{Verbatim}{commandchars=\\\{\}}
% Add ',fontsize=\small' for more characters per line
\usepackage{framed}
\definecolor{shadecolor}{RGB}{248,248,248}
\newenvironment{Shaded}{\begin{snugshade}}{\end{snugshade}}
\newcommand{\KeywordTok}[1]{\textcolor[rgb]{0.13,0.29,0.53}{\textbf{#1}}}
\newcommand{\DataTypeTok}[1]{\textcolor[rgb]{0.13,0.29,0.53}{#1}}
\newcommand{\DecValTok}[1]{\textcolor[rgb]{0.00,0.00,0.81}{#1}}
\newcommand{\BaseNTok}[1]{\textcolor[rgb]{0.00,0.00,0.81}{#1}}
\newcommand{\FloatTok}[1]{\textcolor[rgb]{0.00,0.00,0.81}{#1}}
\newcommand{\ConstantTok}[1]{\textcolor[rgb]{0.00,0.00,0.00}{#1}}
\newcommand{\CharTok}[1]{\textcolor[rgb]{0.31,0.60,0.02}{#1}}
\newcommand{\SpecialCharTok}[1]{\textcolor[rgb]{0.00,0.00,0.00}{#1}}
\newcommand{\StringTok}[1]{\textcolor[rgb]{0.31,0.60,0.02}{#1}}
\newcommand{\VerbatimStringTok}[1]{\textcolor[rgb]{0.31,0.60,0.02}{#1}}
\newcommand{\SpecialStringTok}[1]{\textcolor[rgb]{0.31,0.60,0.02}{#1}}
\newcommand{\ImportTok}[1]{#1}
\newcommand{\CommentTok}[1]{\textcolor[rgb]{0.56,0.35,0.01}{\textit{#1}}}
\newcommand{\DocumentationTok}[1]{\textcolor[rgb]{0.56,0.35,0.01}{\textbf{\textit{#1}}}}
\newcommand{\AnnotationTok}[1]{\textcolor[rgb]{0.56,0.35,0.01}{\textbf{\textit{#1}}}}
\newcommand{\CommentVarTok}[1]{\textcolor[rgb]{0.56,0.35,0.01}{\textbf{\textit{#1}}}}
\newcommand{\OtherTok}[1]{\textcolor[rgb]{0.56,0.35,0.01}{#1}}
\newcommand{\FunctionTok}[1]{\textcolor[rgb]{0.00,0.00,0.00}{#1}}
\newcommand{\VariableTok}[1]{\textcolor[rgb]{0.00,0.00,0.00}{#1}}
\newcommand{\ControlFlowTok}[1]{\textcolor[rgb]{0.13,0.29,0.53}{\textbf{#1}}}
\newcommand{\OperatorTok}[1]{\textcolor[rgb]{0.81,0.36,0.00}{\textbf{#1}}}
\newcommand{\BuiltInTok}[1]{#1}
\newcommand{\ExtensionTok}[1]{#1}
\newcommand{\PreprocessorTok}[1]{\textcolor[rgb]{0.56,0.35,0.01}{\textit{#1}}}
\newcommand{\AttributeTok}[1]{\textcolor[rgb]{0.77,0.63,0.00}{#1}}
\newcommand{\RegionMarkerTok}[1]{#1}
\newcommand{\InformationTok}[1]{\textcolor[rgb]{0.56,0.35,0.01}{\textbf{\textit{#1}}}}
\newcommand{\WarningTok}[1]{\textcolor[rgb]{0.56,0.35,0.01}{\textbf{\textit{#1}}}}
\newcommand{\AlertTok}[1]{\textcolor[rgb]{0.94,0.16,0.16}{#1}}
\newcommand{\ErrorTok}[1]{\textcolor[rgb]{0.64,0.00,0.00}{\textbf{#1}}}
\newcommand{\NormalTok}[1]{#1}
\usepackage{graphicx,grffile}
\makeatletter
\def\maxwidth{\ifdim\Gin@nat@width>\linewidth\linewidth\else\Gin@nat@width\fi}
\def\maxheight{\ifdim\Gin@nat@height>\textheight\textheight\else\Gin@nat@height\fi}
\makeatother
% Scale images if necessary, so that they will not overflow the page
% margins by default, and it is still possible to overwrite the defaults
% using explicit options in \includegraphics[width, height, ...]{}
\setkeys{Gin}{width=\maxwidth,height=\maxheight,keepaspectratio}
\IfFileExists{parskip.sty}{%
\usepackage{parskip}
}{% else
\setlength{\parindent}{0pt}
\setlength{\parskip}{6pt plus 2pt minus 1pt}
}
\setlength{\emergencystretch}{3em}  % prevent overfull lines
\providecommand{\tightlist}{%
  \setlength{\itemsep}{0pt}\setlength{\parskip}{0pt}}
\setcounter{secnumdepth}{0}
% Redefines (sub)paragraphs to behave more like sections
\ifx\paragraph\undefined\else
\let\oldparagraph\paragraph
\renewcommand{\paragraph}[1]{\oldparagraph{#1}\mbox{}}
\fi
\ifx\subparagraph\undefined\else
\let\oldsubparagraph\subparagraph
\renewcommand{\subparagraph}[1]{\oldsubparagraph{#1}\mbox{}}
\fi

%%% Use protect on footnotes to avoid problems with footnotes in titles
\let\rmarkdownfootnote\footnote%
\def\footnote{\protect\rmarkdownfootnote}

%%% Change title format to be more compact
\usepackage{titling}

% Create subtitle command for use in maketitle
\newcommand{\subtitle}[1]{
  \posttitle{
    \begin{center}\large#1\end{center}
    }
}

\setlength{\droptitle}{-2em}

  \title{Assignment 3: Data Exploration}
    \pretitle{\vspace{\droptitle}\centering\huge}
  \posttitle{\par}
    \author{Njeri Kara}
    \preauthor{\centering\large\emph}
  \postauthor{\par}
    \date{}
    \predate{}\postdate{}
  

\begin{document}
\maketitle

\subsection{OVERVIEW}\label{overview}

This exercise accompanies the lessons in Environmental Data Analytics
(ENV872L) on data exploration.

\subsection{Directions}\label{directions}

\begin{enumerate}
\def\labelenumi{\arabic{enumi}.}
\tightlist
\item
  Change ``Student Name'' on line 3 (above) with your name.
\item
  Use the lesson as a guide. It contains code that can be modified to
  complete the assignment.
\item
  Work through the steps, \textbf{creating code and output} that fulfill
  each instruction.
\item
  Be sure to \textbf{answer the questions} in this assignment document.
  Space for your answers is provided in this document and is indicated
  by the ``\textgreater{}'' character. If you need a second paragraph be
  sure to start the first line with ``\textgreater{}''. You should
  notice that the answer is highlighted in green by RStudio.
\item
  When you have completed the assignment, \textbf{Knit} the text and
  code into a single PDF file. You will need to have the correct
  software installed to do this (see Software Installation Guide) Press
  the \texttt{Knit} button in the RStudio scripting panel. This will
  save the PDF output in your Assignments folder.
\item
  After Knitting, please submit the completed exercise (PDF file) to the
  dropbox in Sakai. Please add your last name into the file name (e.g.,
  ``Salk\_A02\_DataExploration.pdf'') prior to submission.
\end{enumerate}

The completed exercise is due on Thursday, 31 January, 2019 before class
begins.

\subsection{1) Set up your R session}\label{set-up-your-r-session}

Check your working directory, load necessary packages (tidyverse), and
upload the North Temperate Lakes long term monitoring dataset for the
light, temperature, and oxygen data for three lakes (file name:
NTL-LTER\_Lake\_ChemistryPhysics\_Raw.csv). Type your code into the R
chunk below.

\begin{Shaded}
\begin{Highlighting}[]
\CommentTok{#Setting the working directory}
\KeywordTok{setwd}\NormalTok{(}\StringTok{"C:/Users/jerik/OneDrive/Documents/Spring 2019 Semenster/Environmental Data Analytics/EDA_R_Work/EDA"}\NormalTok{)}
\CommentTok{#Confirming that it is the correct working directory}
\KeywordTok{getwd}\NormalTok{()}
\end{Highlighting}
\end{Shaded}

\begin{verbatim}
## [1] "C:/Users/jerik/OneDrive/Documents/Spring 2019 Semenster/Environmental Data Analytics/EDA_R_Work/EDA"
\end{verbatim}

\begin{Shaded}
\begin{Highlighting}[]
\CommentTok{#Loading necessary packages}
\KeywordTok{library}\NormalTok{(tidyverse)}
\end{Highlighting}
\end{Shaded}

\begin{verbatim}
## -- Attaching packages --------------------------------------------------------------------------------- tidyverse 1.2.1 --
\end{verbatim}

\begin{verbatim}
## v ggplot2 3.0.0     v purrr   0.2.5
## v tibble  1.4.2     v dplyr   0.7.6
## v tidyr   0.8.2     v stringr 1.3.1
## v readr   1.1.1     v forcats 0.3.0
\end{verbatim}

\begin{verbatim}
## -- Conflicts ------------------------------------------------------------------------------------ tidyverse_conflicts() --
## x dplyr::filter() masks stats::filter()
## x dplyr::lag()    masks stats::lag()
\end{verbatim}

\begin{Shaded}
\begin{Highlighting}[]
\CommentTok{#Uploading the North Temperate Lakes long term monitoring dataset}
\NormalTok{North.Temp.Lakes.data <-}\StringTok{ }\KeywordTok{read.csv}\NormalTok{(}\StringTok{"./Data/Raw/NTL-LTER_Lake_ChemistryPhysics_Raw.csv"}\NormalTok{)}
\end{Highlighting}
\end{Shaded}

\subsection{2) Learn about your system}\label{learn-about-your-system}

Read about your dataset in the NTL-LTER README file. What are three
salient pieces of information you gained from reading this file?

\begin{quote}
ANSWER:
\end{quote}

\begin{quote}
The naming conventions and files format section gives details of the
information that can be derived from the data file name. The data file
is from the database NTL-LTER, the data described is of lakes, details
include chemisty and physics, it is the raw data and it is in csv
format.
\end{quote}

\begin{quote}
The data was accessed, 2018-12-06
\end{quote}

\begin{quote}
The data was assembled by Kateri Salk from the North Temperate Lakes
Long Term Ecological Research website and Kateri Salk can be contacted
at \href{mailto:kateri.salk@duke.edu}{\nolinkurl{kateri.salk@duke.edu}}
for additional information and support.
\end{quote}

\subsection{3) Obtain basic summaries of your
data}\label{obtain-basic-summaries-of-your-data}

Write R commands to display the following information:

\begin{enumerate}
\def\labelenumi{\arabic{enumi}.}
\tightlist
\item
  dimensions of the dataset
\item
  class of the dataset
\item
  first 8 rows of the dataset
\item
  class of the variables lakename, sampledate, depth, and temperature
\item
  summary of lakename, depth, and temperature
\end{enumerate}

\begin{Shaded}
\begin{Highlighting}[]
\CommentTok{# 1 - Dimensions of the dataset}
\KeywordTok{dim}\NormalTok{(North.Temp.Lakes.data) }\CommentTok{#shows the number of rows and columns in the dataset}
\end{Highlighting}
\end{Shaded}

\begin{verbatim}
## [1] 38614    11
\end{verbatim}

\begin{Shaded}
\begin{Highlighting}[]
\CommentTok{# 2 - Class of the dataset}
\KeywordTok{class}\NormalTok{(North.Temp.Lakes.data) }\CommentTok{#type of dataset - data.frame}
\end{Highlighting}
\end{Shaded}

\begin{verbatim}
## [1] "data.frame"
\end{verbatim}

\begin{Shaded}
\begin{Highlighting}[]
\CommentTok{# 3 - First eight rows of the dataset}
\KeywordTok{head}\NormalTok{(North.Temp.Lakes.data,}\DecValTok{8}\NormalTok{) }\CommentTok{#shows first eight rows of the dataset}
\end{Highlighting}
\end{Shaded}

\begin{verbatim}
##   lakeid  lakename year4 daynum sampledate depth temperature_C
## 1      L Paul Lake  1984    148    5/27/84  0.00          14.5
## 2      L Paul Lake  1984    148    5/27/84  0.25            NA
## 3      L Paul Lake  1984    148    5/27/84  0.50            NA
## 4      L Paul Lake  1984    148    5/27/84  0.75            NA
## 5      L Paul Lake  1984    148    5/27/84  1.00          14.5
## 6      L Paul Lake  1984    148    5/27/84  1.50            NA
## 7      L Paul Lake  1984    148    5/27/84  2.00          14.2
## 8      L Paul Lake  1984    148    5/27/84  3.00          11.0
##   dissolvedOxygen irradianceWater irradianceDeck comments
## 1             9.5            1750           1620     <NA>
## 2              NA            1550           1620     <NA>
## 3              NA            1150           1620     <NA>
## 4              NA             975           1620     <NA>
## 5             8.8             870           1620     <NA>
## 6              NA             610           1620     <NA>
## 7             8.6             420           1620     <NA>
## 8            11.5             220           1620     <NA>
\end{verbatim}

\begin{Shaded}
\begin{Highlighting}[]
\CommentTok{# 4}
\KeywordTok{class}\NormalTok{(North.Temp.Lakes.data}\OperatorTok{$}\NormalTok{lakename) }\CommentTok{#class of the variable lakename - factor}
\end{Highlighting}
\end{Shaded}

\begin{verbatim}
## [1] "factor"
\end{verbatim}

\begin{Shaded}
\begin{Highlighting}[]
\KeywordTok{class}\NormalTok{(North.Temp.Lakes.data}\OperatorTok{$}\NormalTok{sampledate) }\CommentTok{#class of the variable sampledate - factor}
\end{Highlighting}
\end{Shaded}

\begin{verbatim}
## [1] "factor"
\end{verbatim}

\begin{Shaded}
\begin{Highlighting}[]
\KeywordTok{class}\NormalTok{(North.Temp.Lakes.data}\OperatorTok{$}\NormalTok{depth) }\CommentTok{#class of the variable depth - numeric}
\end{Highlighting}
\end{Shaded}

\begin{verbatim}
## [1] "numeric"
\end{verbatim}

\begin{Shaded}
\begin{Highlighting}[]
\KeywordTok{class}\NormalTok{(North.Temp.Lakes.data}\OperatorTok{$}\NormalTok{temperature_C) }\CommentTok{#class of the variable temperature_C - numeric}
\end{Highlighting}
\end{Shaded}

\begin{verbatim}
## [1] "numeric"
\end{verbatim}

\begin{Shaded}
\begin{Highlighting}[]
\CommentTok{# 5}
\KeywordTok{summary}\NormalTok{(North.Temp.Lakes.data}\OperatorTok{$}\NormalTok{lakename) }\CommentTok{#summary of the variable lakename}
\end{Highlighting}
\end{Shaded}

\begin{verbatim}
## Central Long Lake     Crampton Lake    East Long Lake  Hummingbird Lake 
##               539              1234              3905               430 
##         Paul Lake        Peter Lake      Tuesday Lake         Ward Lake 
##             10325             11288              6107               598 
##    West Long Lake 
##              4188
\end{verbatim}

\begin{Shaded}
\begin{Highlighting}[]
\KeywordTok{summary}\NormalTok{(North.Temp.Lakes.data}\OperatorTok{$}\NormalTok{depth) }\CommentTok{#summary of the variable depth}
\end{Highlighting}
\end{Shaded}

\begin{verbatim}
##    Min. 1st Qu.  Median    Mean 3rd Qu.    Max. 
##    0.00    1.50    4.00    4.39    6.50   20.00
\end{verbatim}

\begin{Shaded}
\begin{Highlighting}[]
\KeywordTok{summary}\NormalTok{(North.Temp.Lakes.data}\OperatorTok{$}\NormalTok{temperature_C) }\CommentTok{#summary of the variable temerature_C}
\end{Highlighting}
\end{Shaded}

\begin{verbatim}
##    Min. 1st Qu.  Median    Mean 3rd Qu.    Max.    NA's 
##    0.30    5.30    9.30   11.81   18.70   34.10    3858
\end{verbatim}

Change sampledate to class = date. After doing this, write an R command
to display that the class of sammpledate is indeed date. Write another R
command to show the first 10 rows of the date column.

\begin{Shaded}
\begin{Highlighting}[]
\CommentTok{#Checked the North.Temp.Lakes.data dataset to confirm the format of the factor variable sample date }
\CommentTok{#is mm/dd/yy}

\CommentTok{#Changing the sampledate factor variable to a date variable with the format mm/dd/yy}
\NormalTok{North.Temp.Lakes.data}\OperatorTok{$}\NormalTok{sampledate <-}\StringTok{ }\KeywordTok{as.Date}\NormalTok{(North.Temp.Lakes.data}\OperatorTok{$}\NormalTok{sampledate, }\DataTypeTok{format =} \StringTok{"%m/%d/%y"}\NormalTok{)}

\CommentTok{#Confirming that the sampledate variable is a date}
\KeywordTok{class}\NormalTok{(North.Temp.Lakes.data}\OperatorTok{$}\NormalTok{sampledate) }\CommentTok{#new class - Date}
\end{Highlighting}
\end{Shaded}

\begin{verbatim}
## [1] "Date"
\end{verbatim}

Question: Do you want to remove NAs from this dataset? Why or why not?

\begin{Shaded}
\begin{Highlighting}[]
\CommentTok{#summary of all the variables in the dataset to see how many NAs are in each variable has}
\KeywordTok{summary}\NormalTok{(North.Temp.Lakes.data)}
\end{Highlighting}
\end{Shaded}

\begin{verbatim}
##      lakeid                lakename         year4          daynum     
##  R      :11288   Peter Lake    :11288   Min.   :1984   Min.   : 55.0  
##  L      :10325   Paul Lake     :10325   1st Qu.:1991   1st Qu.:166.0  
##  T      : 6107   Tuesday Lake  : 6107   Median :1997   Median :194.0  
##  W      : 4188   West Long Lake: 4188   Mean   :1999   Mean   :194.3  
##  E      : 3905   East Long Lake: 3905   3rd Qu.:2006   3rd Qu.:222.0  
##  M      : 1234   Crampton Lake : 1234   Max.   :2016   Max.   :307.0  
##  (Other): 1567   (Other)       : 1567                                 
##    sampledate             depth       temperature_C   dissolvedOxygen 
##  Min.   :1984-05-27   Min.   : 0.00   Min.   : 0.30   Min.   :  0.00  
##  1st Qu.:1991-08-08   1st Qu.: 1.50   1st Qu.: 5.30   1st Qu.:  0.30  
##  Median :1997-07-28   Median : 4.00   Median : 9.30   Median :  5.60  
##  Mean   :1999-02-05   Mean   : 4.39   Mean   :11.81   Mean   :  4.97  
##  3rd Qu.:2006-06-06   3rd Qu.: 6.50   3rd Qu.:18.70   3rd Qu.:  8.40  
##  Max.   :2016-08-17   Max.   :20.00   Max.   :34.10   Max.   :802.00  
##                                       NA's   :3858    NA's   :4039    
##  irradianceWater     irradianceDeck  
##  Min.   :   -0.337   Min.   :   1.5  
##  1st Qu.:   14.000   1st Qu.: 353.0  
##  Median :   65.000   Median : 747.0  
##  Mean   :  210.242   Mean   : 720.5  
##  3rd Qu.:  265.000   3rd Qu.:1042.0  
##  Max.   :24108.000   Max.   :8532.0  
##  NA's   :14287       NA's   :15419   
##                               comments    
##  DO Probe bad - Doesn't go to zero:  206  
##  DO taken with Jones Lab Meter    :  162  
##  NA's                             :38246  
##                                           
##                                           
##                                           
## 
\end{verbatim}

\begin{Shaded}
\begin{Highlighting}[]
\CommentTok{#Removing rows in the dataset with an NA in the temperature_C variable}
\NormalTok{North.Temp.Lakes.data.no.temp.NAs <-}\StringTok{ }\NormalTok{North.Temp.Lakes.data[}\OperatorTok{!}\KeywordTok{is.na}\NormalTok{(North.Temp.Lakes.data}\OperatorTok{$}\NormalTok{temperature_C), ]}
\CommentTok{#confirming all the rows with an NA in the temperature_C variable have been removed}
\KeywordTok{summary}\NormalTok{(North.Temp.Lakes.data.no.temp.NAs}\OperatorTok{$}\NormalTok{temperature_C) }
\end{Highlighting}
\end{Shaded}

\begin{verbatim}
##    Min. 1st Qu.  Median    Mean 3rd Qu.    Max. 
##    0.30    5.30    9.30   11.81   18.70   34.10
\end{verbatim}

\begin{quote}
ANSWER:
\end{quote}

\begin{quote}
I do not want to remove all the NAs in the dataset. This is because
variables such as comments, irradianceWater and irradianceDeck have a
large proporation of NAs compared to the total number of observations.
If they are to be removed, the dataset rows would significantly reduce
probably impacting data analysis outcomes.
\end{quote}

\begin{quote}
I do however want to remove the rows with NAs in the temperature\_C
variable. These rows are just about 10\% of the total observations and
since temperature\_C is used in all the subsequent plots of question 4,
it may be beneficial for the temperature\_C variable not to have any
missing values.
\end{quote}

\subsection{4) Explore your data
graphically}\label{explore-your-data-graphically}

Write R commands to display graphs depicting:

\begin{enumerate}
\def\labelenumi{\arabic{enumi}.}
\tightlist
\item
  Bar chart of temperature counts for each lake
\item
  Histogram of count distributions of temperature (all temp measurements
  together)
\item
  Change histogram from 2 to have a different number or width of bins
\item
  Frequency polygon of temperature for each lake. Choose different
  colors for each lake.
\item
  Boxplot of temperature for each lake
\item
  Boxplot of temperature based on depth, with depth divided into 0.25 m
  increments
\item
  Scatterplot of temperature by depth
\end{enumerate}

\begin{Shaded}
\begin{Highlighting}[]
\CommentTok{# 1. Bar chart of temperature counts for each lake}
\CommentTok{#The North.Temp.Lakes.data.no.temp.NAs is going to be used because it has only the observations(rows) }
\CommentTok{#with a temperature value. It can therefore be used to plot the number of temperature readings, }
\CommentTok{#temperature count, for each lake}
\KeywordTok{ggplot}\NormalTok{(North.Temp.Lakes.data.no.temp.NAs, }\KeywordTok{aes}\NormalTok{(}\DataTypeTok{x=}\NormalTok{lakename)) }\OperatorTok{+}\StringTok{ }\KeywordTok{geom_bar}\NormalTok{() }\OperatorTok{+}\StringTok{ }
\StringTok{  }\KeywordTok{ggtitle}\NormalTok{(}\StringTok{"Bar chart of temperature counts for each lake"}\NormalTok{) }\OperatorTok{+}\StringTok{ }
\StringTok{  }\KeywordTok{theme}\NormalTok{(}\DataTypeTok{plot.title =} \KeywordTok{element_text}\NormalTok{(}\DataTypeTok{hjust =} \FloatTok{0.5}\NormalTok{)) }\OperatorTok{+}
\StringTok{  }\KeywordTok{xlab}\NormalTok{(}\StringTok{"Lakes"}\NormalTok{) }\OperatorTok{+}\StringTok{ }\KeywordTok{ylab}\NormalTok{(}\StringTok{"Temperature Count"}\NormalTok{) }\CommentTok{#Bar chart with title and labeled x and y axis}
\end{Highlighting}
\end{Shaded}

\includegraphics{Kara_A03_DataExploration_files/figure-latex/unnamed-chunk-4-1.pdf}

\begin{Shaded}
\begin{Highlighting}[]
\CommentTok{# 2. Histogram of count distributions of temperature}
\KeywordTok{ggplot}\NormalTok{(North.Temp.Lakes.data.no.temp.NAs) }\OperatorTok{+}
\StringTok{  }\KeywordTok{geom_histogram}\NormalTok{(}\KeywordTok{aes}\NormalTok{(}\DataTypeTok{x =}\NormalTok{ temperature_C)) }\OperatorTok{+}\StringTok{ }
\StringTok{  }\KeywordTok{ggtitle}\NormalTok{(}\StringTok{"Histogram of count distribution of temperature"}\NormalTok{) }\OperatorTok{+}\StringTok{ }
\StringTok{  }\KeywordTok{theme}\NormalTok{(}\DataTypeTok{plot.title =} \KeywordTok{element_text}\NormalTok{(}\DataTypeTok{hjust =} \FloatTok{0.5}\NormalTok{)) }\CommentTok{#histogram with a title}
\end{Highlighting}
\end{Shaded}

\begin{verbatim}
## `stat_bin()` using `bins = 30`. Pick better value with `binwidth`.
\end{verbatim}

\includegraphics{Kara_A03_DataExploration_files/figure-latex/unnamed-chunk-5-1.pdf}

\begin{Shaded}
\begin{Highlighting}[]
\CommentTok{# 3. Histogram from 2 with a different number of bins}
\KeywordTok{ggplot}\NormalTok{(North.Temp.Lakes.data.no.temp.NAs, }\KeywordTok{aes}\NormalTok{(}\DataTypeTok{x =}\NormalTok{ temperature_C)) }\OperatorTok{+}
\StringTok{  }\KeywordTok{geom_histogram}\NormalTok{(}\DataTypeTok{binwidth =} \DecValTok{1}\NormalTok{) }\OperatorTok{+}\StringTok{ }
\StringTok{  }\KeywordTok{ggtitle}\NormalTok{(}\StringTok{"Histogram of count distributions of temperature with binwidths of 1"}\NormalTok{) }\OperatorTok{+}\StringTok{ }
\StringTok{  }\KeywordTok{theme}\NormalTok{(}\DataTypeTok{plot.title =} \KeywordTok{element_text}\NormalTok{(}\DataTypeTok{hjust =} \FloatTok{0.5}\NormalTok{)) }\CommentTok{#histogram with a binwidth of 1 and a title}
\end{Highlighting}
\end{Shaded}

\includegraphics{Kara_A03_DataExploration_files/figure-latex/unnamed-chunk-6-1.pdf}

\begin{Shaded}
\begin{Highlighting}[]
\CommentTok{# 4. Frequency polygon of temperature for each lake with different colours}
\KeywordTok{ggplot}\NormalTok{(North.Temp.Lakes.data.no.temp.NAs) }\OperatorTok{+}
\StringTok{  }\KeywordTok{geom_freqpoly}\NormalTok{(}\KeywordTok{aes}\NormalTok{(}\DataTypeTok{x =}\NormalTok{ temperature_C, }\DataTypeTok{color =}\NormalTok{ lakename), }\DataTypeTok{bins =} \DecValTok{60}\NormalTok{) }\OperatorTok{+}\StringTok{ }
\StringTok{  }\KeywordTok{theme}\NormalTok{(}\DataTypeTok{legend.position =} \StringTok{"bottom"}\NormalTok{) }\OperatorTok{+}\StringTok{ }\KeywordTok{ggtitle}\NormalTok{(}\StringTok{"Frequency polygon of temperature for each lake"}\NormalTok{) }\OperatorTok{+}\StringTok{ }
\StringTok{  }\KeywordTok{theme}\NormalTok{(}\DataTypeTok{plot.title =} \KeywordTok{element_text}\NormalTok{(}\DataTypeTok{hjust =} \FloatTok{0.5}\NormalTok{)) }\CommentTok{#colour of each line is based on the lakename}
\end{Highlighting}
\end{Shaded}

\includegraphics{Kara_A03_DataExploration_files/figure-latex/unnamed-chunk-7-1.pdf}

\begin{Shaded}
\begin{Highlighting}[]
\CommentTok{#including a legend of the lake colours and a plot title}
\end{Highlighting}
\end{Shaded}

\begin{Shaded}
\begin{Highlighting}[]
\CommentTok{# 5. Boxplot of temperature for each lake}
\KeywordTok{ggplot}\NormalTok{(North.Temp.Lakes.data.no.temp.NAs) }\OperatorTok{+}
\StringTok{  }\KeywordTok{geom_boxplot}\NormalTok{(}\KeywordTok{aes}\NormalTok{(}\DataTypeTok{x =}\NormalTok{ lakename, }\DataTypeTok{y =}\NormalTok{ temperature_C))}\OperatorTok{+}\StringTok{ }
\StringTok{  }\KeywordTok{ggtitle}\NormalTok{(}\StringTok{"Boxplot of temperature for each lake"}\NormalTok{) }\OperatorTok{+}\StringTok{ }\KeywordTok{theme}\NormalTok{(}\DataTypeTok{plot.title =} \KeywordTok{element_text}\NormalTok{(}\DataTypeTok{hjust =} \FloatTok{0.5}\NormalTok{)) }\OperatorTok{+}\StringTok{ }
\StringTok{  }\KeywordTok{xlab}\NormalTok{(}\StringTok{"Lake"}\NormalTok{) }\OperatorTok{+}\StringTok{ }\KeywordTok{ylab}\NormalTok{(}\StringTok{"Temperature_C"}\NormalTok{) }\CommentTok{#including a title and axis labels}
\end{Highlighting}
\end{Shaded}

\includegraphics{Kara_A03_DataExploration_files/figure-latex/unnamed-chunk-8-1.pdf}

\begin{Shaded}
\begin{Highlighting}[]
\CommentTok{# 6. Boxplot of temperature based on depth, with depth divided into 0.25 m increments}
\KeywordTok{ggplot}\NormalTok{(North.Temp.Lakes.data.no.temp.NAs) }\OperatorTok{+}
\StringTok{  }\KeywordTok{geom_boxplot}\NormalTok{(}\KeywordTok{aes}\NormalTok{(}\DataTypeTok{x =}\NormalTok{ depth, }\DataTypeTok{y =}\NormalTok{ temperature_C, }\DataTypeTok{group =} \KeywordTok{cut_width}\NormalTok{(depth, }\FloatTok{0.25}\NormalTok{))) }\OperatorTok{+}\StringTok{ }
\StringTok{  }\KeywordTok{ggtitle}\NormalTok{(}\StringTok{"Boxplot of temperature based on depth"}\NormalTok{) }\OperatorTok{+}\StringTok{ }
\StringTok{  }\KeywordTok{theme}\NormalTok{(}\DataTypeTok{plot.title =} \KeywordTok{element_text}\NormalTok{(}\DataTypeTok{hjust =} \FloatTok{0.5}\NormalTok{)) }\OperatorTok{+}\StringTok{ }\KeywordTok{xlab}\NormalTok{(}\StringTok{"Depth"}\NormalTok{) }\OperatorTok{+}\StringTok{ }
\StringTok{  }\KeywordTok{ylab}\NormalTok{(}\StringTok{"Temperature_C"}\NormalTok{) }\CommentTok{#uncluding a plot title and axis labels}
\end{Highlighting}
\end{Shaded}

\includegraphics{Kara_A03_DataExploration_files/figure-latex/unnamed-chunk-9-1.pdf}

\begin{Shaded}
\begin{Highlighting}[]
\CommentTok{# 7. Scatterplot of temperature by depth}
\KeywordTok{ggplot}\NormalTok{(North.Temp.Lakes.data.no.temp.NAs) }\OperatorTok{+}
\StringTok{  }\KeywordTok{geom_point}\NormalTok{(}\KeywordTok{aes}\NormalTok{(}\DataTypeTok{x =}\NormalTok{ depth, }\DataTypeTok{y =}\NormalTok{ temperature_C)) }\OperatorTok{+}\StringTok{ }\KeywordTok{ggtitle}\NormalTok{(}\StringTok{"Scatterplot of temperature by depth"}\NormalTok{) }\OperatorTok{+}\StringTok{ }
\StringTok{  }\KeywordTok{theme}\NormalTok{(}\DataTypeTok{plot.title =} \KeywordTok{element_text}\NormalTok{(}\DataTypeTok{hjust =} \FloatTok{0.5}\NormalTok{)) }\OperatorTok{+}\StringTok{ }\KeywordTok{xlab}\NormalTok{(}\StringTok{"Depth"}\NormalTok{) }\OperatorTok{+}\StringTok{ }\KeywordTok{ylab}\NormalTok{(}\StringTok{"Temperature_C"}\NormalTok{)}
\end{Highlighting}
\end{Shaded}

\includegraphics{Kara_A03_DataExploration_files/figure-latex/unnamed-chunk-10-1.pdf}
\#\# 5) Form questions for further data analysis

What did you find out about your data from the basic summaries and
graphs you made? Describe in 4-6 sentences.

\begin{quote}
ANSWER:
\end{quote}

\begin{quote}
From the summary of the main dataset I found that only the last 5
variables - temperature\_C,dissolvedOxygen
irradianceWater,irradianceDeck and comments - have missing values and
temperature\_C was missing the least number of variables (3858). I
therfore decided to carry out by subsequent analysis of temperature with
a subset of the dataframe that only had observations with temperature\_c
values.
\end{quote}

\begin{quote}
The histogram and frequency polygons revealed that the temerature data
has a positively skewed distribution. This is consitent with the
temperature's summary statistics because its median is less than its
mean.
\end{quote}

\begin{quote}
The Bar chart and frequency polygon of temperature counts dissaggregate
the temperature count data by lakes clearly showing that Peter Lake has
the highest number of temperature observations followed by Paul lake and
that the most common temperature reading is approximately 5.
\end{quote}

\begin{quote}
The box plot provides a visual break down of the summary statistics of
temperature by lake. It shows for example that the max temperature
reading of 34 was taken at the East Long Lake. It also shows that the
median value of the temperature readings is mainly determinded by
readings from Peter and Paul lakes
\end{quote}

\begin{quote}
The scatterplot and boxplot of temperature based on depth reveal that
the range and number of temperature readings reduces with increasing
depth.
\end{quote}

What are 3 further questions you might ask as you move forward with
analysis of this dataset?

\begin{quote}
ANSWER 1: The number of temperature readings taken every year from 1984
to 2016 and the histortical trend of this temperature data collection by
year.
\end{quote}

\begin{quote}
ANSWER 2: The change in depth observations as years progressed from 1984
to 2016
\end{quote}

\begin{quote}
ANSWER 3: The relationship between obseration depth and the lake the
observation is taken from.
\end{quote}


\end{document}
