\documentclass[]{article}
\usepackage{lmodern}
\usepackage{amssymb,amsmath}
\usepackage{ifxetex,ifluatex}
\usepackage{fixltx2e} % provides \textsubscript
\ifnum 0\ifxetex 1\fi\ifluatex 1\fi=0 % if pdftex
  \usepackage[T1]{fontenc}
  \usepackage[utf8]{inputenc}
\else % if luatex or xelatex
  \ifxetex
    \usepackage{mathspec}
  \else
    \usepackage{fontspec}
  \fi
  \defaultfontfeatures{Ligatures=TeX,Scale=MatchLowercase}
\fi
% use upquote if available, for straight quotes in verbatim environments
\IfFileExists{upquote.sty}{\usepackage{upquote}}{}
% use microtype if available
\IfFileExists{microtype.sty}{%
\usepackage{microtype}
\UseMicrotypeSet[protrusion]{basicmath} % disable protrusion for tt fonts
}{}
\usepackage[margin=2.54cm]{geometry}
\usepackage{hyperref}
\hypersetup{unicode=true,
            pdftitle={Assignment 5: Data Visualization},
            pdfauthor={Njeri Kara},
            pdfborder={0 0 0},
            breaklinks=true}
\urlstyle{same}  % don't use monospace font for urls
\usepackage{color}
\usepackage{fancyvrb}
\newcommand{\VerbBar}{|}
\newcommand{\VERB}{\Verb[commandchars=\\\{\}]}
\DefineVerbatimEnvironment{Highlighting}{Verbatim}{commandchars=\\\{\}}
% Add ',fontsize=\small' for more characters per line
\usepackage{framed}
\definecolor{shadecolor}{RGB}{248,248,248}
\newenvironment{Shaded}{\begin{snugshade}}{\end{snugshade}}
\newcommand{\KeywordTok}[1]{\textcolor[rgb]{0.13,0.29,0.53}{\textbf{#1}}}
\newcommand{\DataTypeTok}[1]{\textcolor[rgb]{0.13,0.29,0.53}{#1}}
\newcommand{\DecValTok}[1]{\textcolor[rgb]{0.00,0.00,0.81}{#1}}
\newcommand{\BaseNTok}[1]{\textcolor[rgb]{0.00,0.00,0.81}{#1}}
\newcommand{\FloatTok}[1]{\textcolor[rgb]{0.00,0.00,0.81}{#1}}
\newcommand{\ConstantTok}[1]{\textcolor[rgb]{0.00,0.00,0.00}{#1}}
\newcommand{\CharTok}[1]{\textcolor[rgb]{0.31,0.60,0.02}{#1}}
\newcommand{\SpecialCharTok}[1]{\textcolor[rgb]{0.00,0.00,0.00}{#1}}
\newcommand{\StringTok}[1]{\textcolor[rgb]{0.31,0.60,0.02}{#1}}
\newcommand{\VerbatimStringTok}[1]{\textcolor[rgb]{0.31,0.60,0.02}{#1}}
\newcommand{\SpecialStringTok}[1]{\textcolor[rgb]{0.31,0.60,0.02}{#1}}
\newcommand{\ImportTok}[1]{#1}
\newcommand{\CommentTok}[1]{\textcolor[rgb]{0.56,0.35,0.01}{\textit{#1}}}
\newcommand{\DocumentationTok}[1]{\textcolor[rgb]{0.56,0.35,0.01}{\textbf{\textit{#1}}}}
\newcommand{\AnnotationTok}[1]{\textcolor[rgb]{0.56,0.35,0.01}{\textbf{\textit{#1}}}}
\newcommand{\CommentVarTok}[1]{\textcolor[rgb]{0.56,0.35,0.01}{\textbf{\textit{#1}}}}
\newcommand{\OtherTok}[1]{\textcolor[rgb]{0.56,0.35,0.01}{#1}}
\newcommand{\FunctionTok}[1]{\textcolor[rgb]{0.00,0.00,0.00}{#1}}
\newcommand{\VariableTok}[1]{\textcolor[rgb]{0.00,0.00,0.00}{#1}}
\newcommand{\ControlFlowTok}[1]{\textcolor[rgb]{0.13,0.29,0.53}{\textbf{#1}}}
\newcommand{\OperatorTok}[1]{\textcolor[rgb]{0.81,0.36,0.00}{\textbf{#1}}}
\newcommand{\BuiltInTok}[1]{#1}
\newcommand{\ExtensionTok}[1]{#1}
\newcommand{\PreprocessorTok}[1]{\textcolor[rgb]{0.56,0.35,0.01}{\textit{#1}}}
\newcommand{\AttributeTok}[1]{\textcolor[rgb]{0.77,0.63,0.00}{#1}}
\newcommand{\RegionMarkerTok}[1]{#1}
\newcommand{\InformationTok}[1]{\textcolor[rgb]{0.56,0.35,0.01}{\textbf{\textit{#1}}}}
\newcommand{\WarningTok}[1]{\textcolor[rgb]{0.56,0.35,0.01}{\textbf{\textit{#1}}}}
\newcommand{\AlertTok}[1]{\textcolor[rgb]{0.94,0.16,0.16}{#1}}
\newcommand{\ErrorTok}[1]{\textcolor[rgb]{0.64,0.00,0.00}{\textbf{#1}}}
\newcommand{\NormalTok}[1]{#1}
\usepackage{graphicx,grffile}
\makeatletter
\def\maxwidth{\ifdim\Gin@nat@width>\linewidth\linewidth\else\Gin@nat@width\fi}
\def\maxheight{\ifdim\Gin@nat@height>\textheight\textheight\else\Gin@nat@height\fi}
\makeatother
% Scale images if necessary, so that they will not overflow the page
% margins by default, and it is still possible to overwrite the defaults
% using explicit options in \includegraphics[width, height, ...]{}
\setkeys{Gin}{width=\maxwidth,height=\maxheight,keepaspectratio}
\IfFileExists{parskip.sty}{%
\usepackage{parskip}
}{% else
\setlength{\parindent}{0pt}
\setlength{\parskip}{6pt plus 2pt minus 1pt}
}
\setlength{\emergencystretch}{3em}  % prevent overfull lines
\providecommand{\tightlist}{%
  \setlength{\itemsep}{0pt}\setlength{\parskip}{0pt}}
\setcounter{secnumdepth}{0}
% Redefines (sub)paragraphs to behave more like sections
\ifx\paragraph\undefined\else
\let\oldparagraph\paragraph
\renewcommand{\paragraph}[1]{\oldparagraph{#1}\mbox{}}
\fi
\ifx\subparagraph\undefined\else
\let\oldsubparagraph\subparagraph
\renewcommand{\subparagraph}[1]{\oldsubparagraph{#1}\mbox{}}
\fi

%%% Use protect on footnotes to avoid problems with footnotes in titles
\let\rmarkdownfootnote\footnote%
\def\footnote{\protect\rmarkdownfootnote}

%%% Change title format to be more compact
\usepackage{titling}

% Create subtitle command for use in maketitle
\newcommand{\subtitle}[1]{
  \posttitle{
    \begin{center}\large#1\end{center}
    }
}

\setlength{\droptitle}{-2em}

  \title{Assignment 5: Data Visualization}
    \pretitle{\vspace{\droptitle}\centering\huge}
  \posttitle{\par}
    \author{Njeri Kara}
    \preauthor{\centering\large\emph}
  \postauthor{\par}
    \date{}
    \predate{}\postdate{}
  

\begin{document}
\maketitle

\subsection{OVERVIEW}\label{overview}

This exercise accompanies the lessons in Environmental Data Analytics
(ENV872L) on data wrangling.

\subsection{Directions}\label{directions}

\begin{enumerate}
\def\labelenumi{\arabic{enumi}.}
\tightlist
\item
  Change ``Student Name'' on line 3 (above) with your name.
\item
  Use the lesson as a guide. It contains code that can be modified to
  complete the assignment.
\item
  Work through the steps, \textbf{creating code and output} that fulfill
  each instruction.
\item
  Be sure to \textbf{answer the questions} in this assignment document.
  Space for your answers is provided in this document and is indicated
  by the ``\textgreater{}'' character. If you need a second paragraph be
  sure to start the first line with ``\textgreater{}''. You should
  notice that the answer is highlighted in green by RStudio.
\item
  When you have completed the assignment, \textbf{Knit} the text and
  code into a single PDF file. You will need to have the correct
  software installed to do this (see Software Installation Guide) Press
  the \texttt{Knit} button in the RStudio scripting panel. This will
  save the PDF output in your Assignments folder.
\item
  After Knitting, please submit the completed exercise (PDF file) to the
  dropbox in Sakai. Please add your last name into the file name (e.g.,
  ``Salk\_A04\_DataWrangling.pdf'') prior to submission.
\end{enumerate}

The completed exercise is due on Tuesday, 19 February, 2019 before class
begins.

\subsection{Set up your session}\label{set-up-your-session}

\begin{enumerate}
\def\labelenumi{\arabic{enumi}.}
\item
  Set up your session. Upload the NTL-LTER processed data files for
  chemistry/physics for Peter and Paul Lakes (tidy and gathered), the
  USGS stream gauge dataset, and the EPA Ecotox dataset for
  Neonicotinoids.
\item
  Make sure R is reading dates as date format, not something else (hint:
  remember that dates were an issue for the USGS gauge data).
\end{enumerate}

\begin{Shaded}
\begin{Highlighting}[]
\CommentTok{#1}
\CommentTok{#Setting the working directory}
\KeywordTok{setwd}\NormalTok{(}\StringTok{"C:/Users/jerik/OneDrive/Documents/Spring 2019 Semenster/Environmental Data Analytics/EDA_R_Work/EDA"}\NormalTok{)}
\CommentTok{#Confirming that it is the correct working directory}
\KeywordTok{getwd}\NormalTok{()}
\end{Highlighting}
\end{Shaded}

\begin{verbatim}
## [1] "C:/Users/jerik/OneDrive/Documents/Spring 2019 Semenster/Environmental Data Analytics/EDA_R_Work/EDA"
\end{verbatim}

\begin{Shaded}
\begin{Highlighting}[]
\CommentTok{#Loading necessary packages}
\KeywordTok{library}\NormalTok{(tidyverse)}
\end{Highlighting}
\end{Shaded}

\begin{verbatim}
## -- Attaching packages -------------------------------------------- tidyverse 1.2.1 --
\end{verbatim}

\begin{verbatim}
## v ggplot2 3.0.0     v purrr   0.2.5
## v tibble  1.4.2     v dplyr   0.7.6
## v tidyr   0.8.2     v stringr 1.3.1
## v readr   1.1.1     v forcats 0.3.0
\end{verbatim}

\begin{verbatim}
## -- Conflicts ----------------------------------------------- tidyverse_conflicts() --
## x dplyr::filter() masks stats::filter()
## x dplyr::lag()    masks stats::lag()
\end{verbatim}

\begin{Shaded}
\begin{Highlighting}[]
\KeywordTok{library}\NormalTok{(lubridate)}
\end{Highlighting}
\end{Shaded}

\begin{verbatim}
## 
## Attaching package: 'lubridate'
\end{verbatim}

\begin{verbatim}
## The following object is masked from 'package:base':
## 
##     date
\end{verbatim}

\begin{Shaded}
\begin{Highlighting}[]
\KeywordTok{library}\NormalTok{(knitr)}
\KeywordTok{library}\NormalTok{(gridExtra)}
\end{Highlighting}
\end{Shaded}

\begin{verbatim}
## 
## Attaching package: 'gridExtra'
\end{verbatim}

\begin{verbatim}
## The following object is masked from 'package:dplyr':
## 
##     combine
\end{verbatim}

\begin{Shaded}
\begin{Highlighting}[]
\CommentTok{#Uploading the required dataset.}
\NormalTok{NTL.Nutrients.PP.gathered.process.D <-}
\StringTok{  }\KeywordTok{read.csv}\NormalTok{(}\StringTok{"./Data/Processed/NTL-LTER_Lake_Nutrients_PeterPaulGathered_Processed.csv"}\NormalTok{)}

\NormalTok{NTL.Nutrients.PP.process.D <-}
\StringTok{  }\KeywordTok{read.csv}\NormalTok{(}\StringTok{"./Data/Processed/NTL-LTER_Lake_Nutrients_PeterPaul_Processed.csv"}\NormalTok{)}

\NormalTok{ECOTOX.Neonicotinoids.Mortality.raw.D <-}\StringTok{ }
\StringTok{  }\KeywordTok{read.csv}\NormalTok{(}\StringTok{"./Data/Raw/ECOTOX_Neonicotinoids_Mortality_raw.csv"}\NormalTok{)}

\NormalTok{USGS.Flow.raw.D <-}\StringTok{ }
\StringTok{  }\KeywordTok{read.csv}\NormalTok{(}\StringTok{"./Data/Raw/USGS_Site02085000_Flow_Raw.csv"}\NormalTok{)}

\CommentTok{#Exploring the datasets to determine data columns and format}
\NormalTok{str}
\end{Highlighting}
\end{Shaded}

\begin{verbatim}
## function (object, ...) 
## UseMethod("str")
## <bytecode: 0x00000000185891e0>
## <environment: namespace:utils>
\end{verbatim}

\begin{Shaded}
\begin{Highlighting}[]
\CommentTok{#2}
\CommentTok{#Exploring the datasets to determine data columns and format; }
\CommentTok{#changing date variable to date format}
\KeywordTok{str}\NormalTok{(NTL.Nutrients.PP.process.D)}
\end{Highlighting}
\end{Shaded}

\begin{verbatim}
## 'data.frame':    2770 obs. of  13 variables:
##  $ lakeid    : Factor w/ 2 levels "L","R": 1 1 1 1 1 1 2 2 2 2 ...
##  $ lakename  : Factor w/ 2 levels "Paul Lake","Peter Lake": 1 1 1 1 1 1 2 2 2 2 ...
##  $ year4     : int  1991 1991 1991 1991 1991 1991 1991 1991 1991 1991 ...
##  $ daynum    : int  140 140 140 140 140 140 140 140 140 140 ...
##  $ sampledate: Factor w/ 778 levels "1991-05-20","1991-05-27",..: 1 1 1 1 1 1 1 1 1 1 ...
##  $ depth_id  : int  1 2 3 4 5 6 1 2 3 4 ...
##  $ depth     : num  0 0.85 1.75 3 4 6 0 1 2.25 3.5 ...
##  $ tn_ug     : num  538 285 399 453 363 583 352 356 364 582 ...
##  $ tp_ug     : num  25 14 14 14 13 37 11 15 28 14 ...
##  $ nh34      : num  NA NA NA NA NA NA NA NA NA NA ...
##  $ no23      : num  NA NA NA NA NA NA NA NA NA NA ...
##  $ po4       : num  NA NA NA NA NA NA NA NA NA NA ...
##  $ comments  : logi  NA NA NA NA NA NA ...
\end{verbatim}

\begin{Shaded}
\begin{Highlighting}[]
\NormalTok{NTL.Nutrients.PP.process.D}\OperatorTok{$}\NormalTok{sampledate <-}\StringTok{ }\KeywordTok{as.Date}\NormalTok{(NTL.Nutrients.PP.process.D}\OperatorTok{$}\NormalTok{sampledate, }\DataTypeTok{format =} \StringTok{"%Y-%m-%d"}\NormalTok{)}

\KeywordTok{str}\NormalTok{(NTL.Nutrients.PP.gathered.process.D)}
\end{Highlighting}
\end{Shaded}

\begin{verbatim}
## 'data.frame':    7997 obs. of  7 variables:
##  $ lakename     : Factor w/ 2 levels "Paul Lake","Peter Lake": 1 1 1 1 1 1 2 2 2 2 ...
##  $ daynum       : int  140 140 140 140 140 140 140 140 140 140 ...
##  $ year4        : int  1991 1991 1991 1991 1991 1991 1991 1991 1991 1991 ...
##  $ sampledate   : Factor w/ 778 levels "1991-05-20","1991-05-27",..: 1 1 1 1 1 1 1 1 1 1 ...
##  $ depth        : num  0 0.85 1.75 3 4 6 0 1 2.25 3.5 ...
##  $ nutrient     : Factor w/ 5 levels "nh34","no23",..: 4 4 4 4 4 4 4 4 4 4 ...
##  $ concentration: num  538 285 399 453 363 583 352 356 364 582 ...
\end{verbatim}

\begin{Shaded}
\begin{Highlighting}[]
\NormalTok{NTL.Nutrients.PP.gathered.process.D}\OperatorTok{$}\NormalTok{sampledate <-}\StringTok{ }\KeywordTok{as.Date}\NormalTok{(NTL.Nutrients.PP.gathered.process.D}\OperatorTok{$}\NormalTok{sampledate, }\DataTypeTok{format =} \StringTok{"%Y-%m-%d"}\NormalTok{)}


\KeywordTok{str}\NormalTok{(USGS.Flow.raw.D)}
\end{Highlighting}
\end{Shaded}

\begin{verbatim}
## 'data.frame':    33216 obs. of  15 variables:
##  $ agency_cd             : Factor w/ 1 level "USGS": 1 1 1 1 1 1 1 1 1 1 ...
##  $ site_no               : int  2085000 2085000 2085000 2085000 2085000 2085000 2085000 2085000 2085000 2085000 ...
##  $ datetime              : Factor w/ 33216 levels "1/1/00","1/1/01",..: 20 1021 2022 2295 2386 2477 2568 2659 2750 111 ...
##  $ X165986_00060_00001   : num  74 61 56 54 48 47 44 41 44 57 ...
##  $ X165986_00060_00001_cd: Factor w/ 4 levels "","A","A:e","P": 2 2 2 2 2 2 2 2 2 2 ...
##  $ X165987_00060_00002   : num  NA NA NA NA NA NA NA NA NA NA ...
##  $ X165987_00060_00002_cd: Factor w/ 3 levels "","A","P": 1 1 1 1 1 1 1 1 1 1 ...
##  $ X84936_00060_00003    : num  NA NA NA NA NA NA NA NA NA NA ...
##  $ X84936_00060_00003_cd : Factor w/ 3 levels "","A","P": 1 1 1 1 1 1 1 1 1 1 ...
##  $ X84937_00065_00001    : num  NA NA NA NA NA NA NA NA NA NA ...
##  $ X84937_00065_00001_cd : Factor w/ 3 levels "","A","P": 1 1 1 1 1 1 1 1 1 1 ...
##  $ X84938_00065_00002    : num  NA NA NA NA NA NA NA NA NA NA ...
##  $ X84938_00065_00002_cd : Factor w/ 3 levels "","A","P": 1 1 1 1 1 1 1 1 1 1 ...
##  $ X84939_00065_00003    : num  NA NA NA NA NA NA NA NA NA NA ...
##  $ X84939_00065_00003_cd : Factor w/ 3 levels "","A","P": 1 1 1 1 1 1 1 1 1 1 ...
\end{verbatim}

\begin{Shaded}
\begin{Highlighting}[]
\CommentTok{#Changing the class of the datetime variable to date}
\NormalTok{USGS.Flow.raw.D}\OperatorTok{$}\NormalTok{datetime <-}\StringTok{ }\KeywordTok{as.Date}\NormalTok{(USGS.Flow.raw.D}\OperatorTok{$}\NormalTok{datetime, }\DataTypeTok{format =} \StringTok{"%m/%d/%y"}\NormalTok{)}
\CommentTok{#changing the date format to be consistent with the other datasets}
\NormalTok{USGS.Flow.raw.D}\OperatorTok{$}\NormalTok{datetime <-}\StringTok{ }\KeywordTok{format}\NormalTok{(USGS.Flow.raw.D}\OperatorTok{$}\NormalTok{datetime, }\StringTok{"%y%m%d"}\NormalTok{)}
\CommentTok{#correcting error in }
\CommentTok{#creating a function that specifies that if d is greater than 181231 (%y%m%d - format) then date should be 19 and if not then it should be 20; and then print d}
\NormalTok{date.correction.func <-}\StringTok{ }\NormalTok{(}\ControlFlowTok{function}\NormalTok{(d) \{}
       \KeywordTok{paste0}\NormalTok{(}\KeywordTok{ifelse}\NormalTok{(d }\OperatorTok{>}\StringTok{ }\DecValTok{181231}\NormalTok{,}\StringTok{"19"}\NormalTok{,}\StringTok{"20"}\NormalTok{),d)}
\NormalTok{       \})}
\CommentTok{#running the created function with d as datetime for the dataset USGS.flow.data}
\NormalTok{USGS.Flow.raw.D}\OperatorTok{$}\NormalTok{datetime <-}\StringTok{ }\KeywordTok{date.correction.func}\NormalTok{(USGS.Flow.raw.D}\OperatorTok{$}\NormalTok{datetime)}

\CommentTok{# formating the created datetime as a date}
\NormalTok{USGS.Flow.raw.D}\OperatorTok{$}\NormalTok{datetime <-}\StringTok{ }\KeywordTok{as.Date}\NormalTok{(USGS.Flow.raw.D}\OperatorTok{$}\NormalTok{datetime, }\DataTypeTok{format =} \StringTok{"%Y%m%d"}\NormalTok{)}
\end{Highlighting}
\end{Shaded}

\subsection{Define your theme}\label{define-your-theme}

\begin{enumerate}
\def\labelenumi{\arabic{enumi}.}
\setcounter{enumi}{2}
\tightlist
\item
  Build a theme and set it as your default theme.
\end{enumerate}

\begin{Shaded}
\begin{Highlighting}[]
\CommentTok{#3}
\CommentTok{#building a theme}
\NormalTok{NK.theme <-}\StringTok{ }\KeywordTok{theme_light}\NormalTok{(}\DataTypeTok{base_size =} \DecValTok{12}\NormalTok{) }\OperatorTok{+}
\StringTok{  }\KeywordTok{theme}\NormalTok{(}\DataTypeTok{plot.background =} \KeywordTok{element_rect}\NormalTok{(}\DataTypeTok{fill =} \StringTok{"grey97"}\NormalTok{),}\DataTypeTok{panel.grid.major =}\KeywordTok{element_line}\NormalTok{(}\DataTypeTok{linetype =} \StringTok{"dotted"}\NormalTok{),}\DataTypeTok{panel.grid.minor =} \KeywordTok{element_line}\NormalTok{(}\DataTypeTok{linetype =} \StringTok{"dotted"}\NormalTok{), }\DataTypeTok{text=}\KeywordTok{element_text}\NormalTok{(}\DataTypeTok{size =} \DecValTok{14}\NormalTok{, }\DataTypeTok{color =} \StringTok{"black"}\NormalTok{, }\DataTypeTok{face =} \StringTok{"bold"}\NormalTok{),}\DataTypeTok{axis.text =} \KeywordTok{element_text}\NormalTok{(}\DataTypeTok{color =} \StringTok{"grey40"}\NormalTok{), }\DataTypeTok{legend.position =} \StringTok{"right"}\NormalTok{,}\DataTypeTok{legend.text =} \KeywordTok{element_text}\NormalTok{(}\DataTypeTok{color =} \StringTok{"grey40"}\NormalTok{))}

\CommentTok{#setting it as my default theme}
\KeywordTok{theme_set}\NormalTok{(NK.theme)}
\end{Highlighting}
\end{Shaded}

\subsection{Create graphs}\label{create-graphs}

For numbers 4-7, create graphs that follow best practices for data
visualization. To make your graphs ``pretty,'' ensure your theme, color
palettes, axes, and legends are edited to your liking.

Hint: a good way to build graphs is to make them ugly first and then
create more code to make them pretty.

\begin{enumerate}
\def\labelenumi{\arabic{enumi}.}
\setcounter{enumi}{3}
\tightlist
\item
  {[}NTL-LTER{]} Plot total phosphorus by phosphate, with separate
  aesthetics for Peter and Paul lakes. Add a line of best fit and color
  it black.
\end{enumerate}

\begin{Shaded}
\begin{Highlighting}[]
\CommentTok{#4}
\NormalTok{NTL.LTER.Plot.Q4 <-}\StringTok{ }\KeywordTok{ggplot}\NormalTok{(NTL.Nutrients.PP.process.D, }\KeywordTok{aes}\NormalTok{(}\DataTypeTok{x =}\NormalTok{ tp_ug, }\DataTypeTok{y =}\NormalTok{ po4, }\DataTypeTok{color =}\NormalTok{ lakename)) }\OperatorTok{+}
\StringTok{  }\KeywordTok{geom_point}\NormalTok{() }\OperatorTok{+}
\StringTok{  }\KeywordTok{geom_smooth}\NormalTok{(}\DataTypeTok{method =}\NormalTok{ lm, }\DataTypeTok{color =} \StringTok{"black"}\NormalTok{) }\OperatorTok{+}
\StringTok{  }\KeywordTok{scale_color_manual}\NormalTok{(}\DataTypeTok{values =} \KeywordTok{c}\NormalTok{(}\StringTok{"#f1a340"}\NormalTok{, }\StringTok{"#998ec3"}\NormalTok{)) }\OperatorTok{+}
\StringTok{  }\KeywordTok{ylim}\NormalTok{(}\KeywordTok{c}\NormalTok{(}\DecValTok{0}\NormalTok{,}\DecValTok{45}\NormalTok{)) }\OperatorTok{+}
\StringTok{  }\KeywordTok{ggtitle}\NormalTok{(}\StringTok{"Q4. Plot of total phosphorus by phosphate"}\NormalTok{) }\OperatorTok{+}
\StringTok{  }\KeywordTok{xlab}\NormalTok{(}\StringTok{"Total phosphorous (\textbackslash{}U003BCg/L)"}\NormalTok{) }\OperatorTok{+}
\StringTok{  }\KeywordTok{ylab}\NormalTok{(}\KeywordTok{expression}\NormalTok{(}\StringTok{"Phosphate ( PO"}\NormalTok{[}\DecValTok{4}\NormalTok{]}\OperatorTok{*}\StringTok{ " )"}\NormalTok{))}

\KeywordTok{print}\NormalTok{(NTL.LTER.Plot.Q4)}
\end{Highlighting}
\end{Shaded}

\begin{verbatim}
## Warning: Removed 1708 rows containing non-finite values (stat_smooth).
\end{verbatim}

\begin{verbatim}
## Warning: Removed 1708 rows containing missing values (geom_point).
\end{verbatim}

\begin{verbatim}
## Warning: Removed 2 rows containing missing values (geom_smooth).
\end{verbatim}

\includegraphics{Kara_A05_DataVisualization_files/figure-latex/unnamed-chunk-3-1.pdf}

\begin{enumerate}
\def\labelenumi{\arabic{enumi}.}
\setcounter{enumi}{4}
\tightlist
\item
  {[}NTL-LTER{]} Plot nutrients by date for Peter Lake, with separate
  colors for each depth. Facet your graph by the nutrient type.
\end{enumerate}

\begin{Shaded}
\begin{Highlighting}[]
\CommentTok{#5}
\CommentTok{#geting a subset of data for only peter lake}
\NormalTok{NTL.LTER.Peter.Q5.data <-}\StringTok{ }\KeywordTok{subset}\NormalTok{(NTL.Nutrients.PP.gathered.process.D, lakename }\OperatorTok{==}\StringTok{"Peter Lake"}\NormalTok{)}
\CommentTok{#ploting peter lake data}
\NormalTok{NTL.LTER.Peter.plot.Q5  <-}
\StringTok{  }\KeywordTok{ggplot}\NormalTok{(NTL.LTER.Peter.Q5.data, }\KeywordTok{aes}\NormalTok{(}\DataTypeTok{x =}\NormalTok{ sampledate, }\DataTypeTok{y =}\NormalTok{ concentration , }\DataTypeTok{colour =}\NormalTok{ depth, }\DataTypeTok{fill =}\NormalTok{ depth)) }\OperatorTok{+}
\StringTok{  }\KeywordTok{geom_point}\NormalTok{(}\DataTypeTok{size =} \DecValTok{2}\NormalTok{) }\OperatorTok{+}
\StringTok{  }\KeywordTok{facet_wrap}\NormalTok{(}\KeywordTok{vars}\NormalTok{(nutrient), }\DataTypeTok{nrow =} \DecValTok{5}\NormalTok{, }\DataTypeTok{scales =} \StringTok{"free_y"}\NormalTok{) }\OperatorTok{+}
\StringTok{  }\KeywordTok{theme}\NormalTok{(}\DataTypeTok{strip.background =} \KeywordTok{element_rect}\NormalTok{(}\DataTypeTok{fill =} \StringTok{"black"}\NormalTok{), }\DataTypeTok{strip.text =} \KeywordTok{element_text}\NormalTok{(}\DataTypeTok{color =} \StringTok{"white"}\NormalTok{)) }\OperatorTok{+}\StringTok{ }
\StringTok{  }\KeywordTok{ggtitle}\NormalTok{(}\StringTok{"Q5. Plots by nutrients by date for Peter Lake"}\NormalTok{) }\OperatorTok{+}
\StringTok{  }\KeywordTok{xlab}\NormalTok{(}\StringTok{"Date (years)"}\NormalTok{) }\OperatorTok{+}
\StringTok{  }\KeywordTok{ylab}\NormalTok{(}\StringTok{"Nutrient concentration"}\NormalTok{)}

\KeywordTok{print}\NormalTok{(NTL.LTER.Peter.plot.Q5)}
\end{Highlighting}
\end{Shaded}

\includegraphics{Kara_A05_DataVisualization_files/figure-latex/unnamed-chunk-4-1.pdf}

\begin{enumerate}
\def\labelenumi{\arabic{enumi}.}
\setcounter{enumi}{5}
\tightlist
\item
  {[}USGS gauge{]} Plot discharge by date. Create two plots, one with
  the points connected with geom\_line and one with the points connected
  with geom\_smooth (hint: do not use method = ``lm''). Place these
  graphs on the same plot (hint: ggarrange or something similar)
\end{enumerate}

\begin{Shaded}
\begin{Highlighting}[]
\CommentTok{#6}
\CommentTok{#plot with geom_line}
\NormalTok{USGS.plot.Q6.line <-}\StringTok{ }\KeywordTok{ggplot}\NormalTok{(USGS.Flow.raw.D, }\KeywordTok{aes}\NormalTok{(}\DataTypeTok{x =}\NormalTok{ datetime, }\DataTypeTok{y =}\NormalTok{X165986_00060_}\DecValTok{00001}\NormalTok{)) }\OperatorTok{+}\StringTok{ }
\StringTok{  }\KeywordTok{geom_line}\NormalTok{(}\DataTypeTok{color =} \StringTok{"darkblue"}\NormalTok{) }\OperatorTok{+}\StringTok{ }
\StringTok{  }\KeywordTok{ggtitle}\NormalTok{(}\StringTok{"Q6a. Geom_line plot of discharge by date"}\NormalTok{) }\OperatorTok{+}
\StringTok{  }\KeywordTok{xlab}\NormalTok{(}\StringTok{"Date (years)"}\NormalTok{) }\OperatorTok{+}
\StringTok{  }\KeywordTok{ylab}\NormalTok{(}\StringTok{"Discharge"}\NormalTok{) }

\CommentTok{#plot with geom_smooth}
\NormalTok{USGS.plot.Q6.smooth <-}\StringTok{ }\KeywordTok{ggplot}\NormalTok{(USGS.Flow.raw.D, }\KeywordTok{aes}\NormalTok{(}\DataTypeTok{x =}\NormalTok{ datetime, }\DataTypeTok{y =}\NormalTok{X165986_00060_}\DecValTok{00001}\NormalTok{)) }\OperatorTok{+}
\StringTok{  }\KeywordTok{geom_smooth}\NormalTok{(}\DataTypeTok{color =} \StringTok{"darkblue"}\NormalTok{) }\OperatorTok{+}\StringTok{ }
\StringTok{  }\KeywordTok{ggtitle}\NormalTok{(}\StringTok{"Q6b. Geom_smooth plot of discharge by date"}\NormalTok{) }\OperatorTok{+}
\StringTok{  }\KeywordTok{xlab}\NormalTok{(}\StringTok{"Date (years)"}\NormalTok{) }\OperatorTok{+}
\StringTok{  }\KeywordTok{ylab}\NormalTok{(}\StringTok{"Discharge"}\NormalTok{)}

\CommentTok{#placing both graphs on the same plot}
\KeywordTok{grid.arrange}\NormalTok{(USGS.plot.Q6.line, USGS.plot.Q6.smooth)}
\end{Highlighting}
\end{Shaded}

\begin{verbatim}
## `geom_smooth()` using method = 'gam' and formula 'y ~ s(x, bs = "cs")'
\end{verbatim}

\begin{verbatim}
## Warning: Removed 5113 rows containing non-finite values (stat_smooth).
\end{verbatim}

\includegraphics{Kara_A05_DataVisualization_files/figure-latex/unnamed-chunk-5-1.pdf}
Question: How do these two types of lines affect your interpretation of
the data?

\begin{quote}
Answer: Yes. The geom\_line plot shows significant fluctuations in data
across the years.The data range appears to be from 0 to about 5,000. It
also clearly shows missing data in 1980. Data across years does not seam
to have a distinguishable trend.
\end{quote}

\begin{quote}
The geom\_smooth plot on the other hand does show as much variation in
data. The data range appears to be from about 25 to 125. It also does
not show any indication of missing data and it change in data across
years appears to have a trend.
\end{quote}

\begin{enumerate}
\def\labelenumi{\arabic{enumi}.}
\setcounter{enumi}{6}
\tightlist
\item
  {[}ECOTOX Neonicotinoids{]} Plot the concentration, divided by
  chemical name. Choose a geom that accurately portrays the distribution
  of data points.
\end{enumerate}

\begin{Shaded}
\begin{Highlighting}[]
\CommentTok{#7}
\CommentTok{#selectring subset of data for concentrations in mg/L}
\NormalTok{Ecotox.plot.Q7 <-}\StringTok{ }\KeywordTok{subset}\NormalTok{(ECOTOX.Neonicotinoids.Mortality.raw.D, Conc..Units..Std. }\OperatorTok{==}\StringTok{"AI mg/L"}\NormalTok{)}

\CommentTok{#plotting graph}
\NormalTok{Ecotox.plot.Q7 <-}\StringTok{ }\KeywordTok{ggplot}\NormalTok{(ECOTOX.Neonicotinoids.Mortality.raw.D, }\KeywordTok{aes}\NormalTok{(}\DataTypeTok{x =}\NormalTok{ Chemical.Name, }\DataTypeTok{y =}\NormalTok{ Conc..Mean..Std.)) }\OperatorTok{+}\StringTok{ }
\StringTok{  }\KeywordTok{geom_boxplot}\NormalTok{(}\KeywordTok{aes}\NormalTok{(}\DataTypeTok{color =}\NormalTok{ Chemical.Name)) }\OperatorTok{+}
\StringTok{  }\KeywordTok{ylim}\NormalTok{(}\KeywordTok{c}\NormalTok{(}\DecValTok{0}\NormalTok{,}\DecValTok{750}\NormalTok{)) }\OperatorTok{+}
\StringTok{  }\KeywordTok{coord_flip}\NormalTok{() }\OperatorTok{+}\StringTok{ }
\StringTok{  }\KeywordTok{theme}\NormalTok{(}\DataTypeTok{legend.position =} \StringTok{"none"}\NormalTok{) }\OperatorTok{+}
\StringTok{  }\KeywordTok{ggtitle}\NormalTok{(}\StringTok{"Q7. Plot of concentration by chemical name"}\NormalTok{) }\OperatorTok{+}\StringTok{ }
\StringTok{  }\KeywordTok{xlab}\NormalTok{(}\StringTok{"Chemical name"}\NormalTok{) }\OperatorTok{+}
\StringTok{  }\KeywordTok{ylab}\NormalTok{(}\StringTok{"Concentration"}\NormalTok{)}

\KeywordTok{print}\NormalTok{(Ecotox.plot.Q7)}
\end{Highlighting}
\end{Shaded}

\begin{verbatim}
## Warning: Removed 77 rows containing non-finite values (stat_boxplot).
\end{verbatim}

\includegraphics{Kara_A05_DataVisualization_files/figure-latex/unnamed-chunk-6-1.pdf}


\end{document}
